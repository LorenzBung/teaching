\documentclass[11pt, a4paper, oneside]{article}
\usepackage[ngerman]{babel}
\usepackage{worksheet}

\begin{document}
	\author{L. Bung}
	\title{Definitions- und Wertebereich}
	\subject{Mathematik}
	\class{FHR1}
	\date{05.11.2025}
	\maketitle
	
	\partnertask[(\timeLimit{12 min.})]{Schwierige Funktion}
	
	Untersuchen Sie gemeinsam die Funktion $\displaystyle f(x) = \frac{1}{x-3} + \sqrt{x+1}.$
	
	\begin{itemize}
		\item Gibt es x-Werte, die man nicht in die Funktion einsetzen kann?
		Falls ja: Warum?
		\item Welche y-Werte können herauskommen? Welche nicht?
	\end{itemize}
	
	\begin{plot}[xmin=-3, xmax=9, ymin=-3, ymax=6]
		\addplot[domain=-1.1:-0.9, samples=100, color=red, thick]{1/(x-3) + sqrt(x+1)};
		\addplot[domain=-.9:2.9, samples=100, color=red, thick]{1/(x-3) + sqrt(x+1)};
		\addplot[domain=3.1:9, samples=100, color=red, thick]{1/(x-3) + sqrt(x+1)} node[above left]{$f(x)$};
	\end{plot}
	
	\checkered[6cm]
	
	\hint{Definitions- und Wertebereich einer Funktion}{
		\begin{figure}[H]
			\centering
			\begin{tikzpicture}
				\draw (0,0) ellipse (1.5cm and 2cm);
				\node at (0,-2.5) {Definitionsbereich $\mathbb{D}_f$};
				\draw (5,0) ellipse (1.5cm and 2cm);
				\node at (5,-2.5) {Wertebereich $\mathbb{W}_f$};
				\path[->,>=stealth] (1,0) edge[bend left=30] node[above]{Funktion $f$} (4,0);
			\end{tikzpicture}
		\end{figure}
		
		\textbf{Definitionsbereich}:
		
		\vspace{.75cm}
		\textbf{Wertebereich}:
		
		\vspace{.75cm}
		\textbf{Definitionslücke}:
		\vspace{.75cm}
	}
	
	\singletask[(\timeLimit{7 min.})]{Funktionen (unter)suchen}
	
	a) Geben Sie den Definitions- und Wertebereich der Funktion $f(x) = 2x^2+1$ an.
	
	b) Finden Sie eine Funktion, die den Definitionsbereich $\mathbb{D} = \mathbb{R} \setminus \left\{-2\right\}$ hat.
	
	\checkered[5cm]
	
	\vspace{-1cm}
	
	\begin{wrapfigure}{r}{.23\textwidth}
		\vspace{.25cm}
		\qrlink{https://learningapps.org/watch?v=pozya2d8a25}
	\end{wrapfigure}
	
	\partnertask[(\timeLimit{7 min.})]{Zuordnung}
	
	Bearbeiten Sie die LearningApp.
\end{document}