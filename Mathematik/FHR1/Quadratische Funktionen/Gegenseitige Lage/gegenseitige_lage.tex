\documentclass[11pt, a4paper, oneside]{article}
\usepackage[ngerman]{babel}
\usepackage{worksheet}

\begin{document}
	\author{L. Bung}
	\title{Lage von Gerade/Parabel \hspace{10cm} und Parabel/Parabel}
	\subject{Mathematik}
	\class{FHR1}
	\maketitle
	
	\partnertask{Mögliche Lagen linearer und quadratischer Funktionen}
	
	a) Überlegen Sie zusammen, welche möglichen Lagen eine Gerade $g(x)$ und eine Parabel $p(x)$ zueinander haben können.
	Fertigen Sie Skizzen an.
	Wie viele Schnittpunkte haben Sie in diesen Fällen jeweils?
	
	\checkered[7cm]
	
	\hint{Gegenseitige Lage von Gerade und Parabel}{
		\phantom{x}\vspace{7cm}
	}
	
	b) Überlegen Sie jetzt, welche Möglichkeiten es für zwei Parabeln $p(x)$ und $q(x)$ gibt.
	Machen Sie auch hier Skizzen.
	Wie viele Schnittpunkte gibt es hier jeweils?
	
	\checkered[9cm]
	
	\hint{Gegenseitige Lage von Parabel und Parabel}{
		\phantom{x}\vspace{10cm}
	}
	
	\singletask{Schnittpunkte berechnen}
	
	Bestimmen Sie jeweils den bzw. die Schnittpunkt(e) der beiden Funktionen:
	
	\begin{enumerate}[label=\alph*)]
		\item $f(x) = x-\frac{1}{2}$ und $g(x) = \frac{1}{2}x^2$
		\item $f(x) = 2(x-1)^2-1$ und $g(x) = 2(x-2)^2+1$
		\item $f(x) = 10x^2+1$ und $g(x) = -10x^2-1$
	\end{enumerate}
	
	\checkered[17cm]
	
	\singletask{Passende Funktionen zu gegebenen Eigenschaften finden}
	
	Finden Sie eine quadratische Funktion $g(x)$, die...
	
	\begin{enumerate}[label=\alph*)]
		\item ...einen Schnittpunkt mit der Funktion $f(x) = 2x+1$ hat.
		\item ...zwei Schnittpunkte mit der Funktion $f(x) = 2(x+2)^2-3$ hat.
	\end{enumerate}
	
	Geben Sie jeweils den bzw. die Schnittpunkt(e) der beiden Funktionen an.
	
	\checkered[11cm]
	
	\begin{wrapfigure}{r}{.3\textwidth}
		\centering
		\qrlink{https://create.kahoot.it/share/schnittpunkte-von-funktionen-hochstens-zweiten-grades/b8ae1424-4184-48a2-887e-957b62792c57}
	\end{wrapfigure}
	
	\grouptask{Kahoot}
	
	Spielen Sie gemeinsam das Kahoot.
	
	Überprüfen Sie dabei, wo Sie Verständnisschwierigkeiten haben und arbeiten Sie diese anschließend auf.
\end{document}