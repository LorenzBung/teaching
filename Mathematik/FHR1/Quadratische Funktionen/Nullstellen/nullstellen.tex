\documentclass[11pt, a4paper, oneside]{article}
\usepackage[ngerman]{babel}
\usepackage{worksheet}

\begin{document}
	\author{L. Bung}
	\title{Nullstellen quadratischer Funktionen}
	\subject{Mathematik}
	\class{FHR1}
	\maketitle
	
	\task{Nullstellen in Produktform}
	
	a) Berechnen Sie die Nullstellen der folgenden Funktionen mithilfe der abc- oder pq-Formel.
	Bringen Sie die Funktionsgleichungen dazu zuerst in Normalform.
	
	\begin{enumerate}[label=]
		\item $f(x) = (x-1)(x-3)=0$
		\item $g(x) = 4(x+5)(x-4)=0$
		\item $h(x) = 6x(x-5)=0$
		\item $i(x) = (x+1)^2=0$
	\end{enumerate}
	
	\checkered[9.5cm]
	
	b) Überprüfen Sie Ihre Ergebnisse mit Ihrem Sitznachbarn.
	Was fällt Ihnen auf, wenn Sie die Nullstellen mit der Funktionsgleichung vergleichen?
	
	\lines[3cm]
	
	\hint{Produktform}{
		\phantom{x}\vspace{7cm}
	}
	
	\singletask{Funktionen für gegebene Nullstellen finden}
	
	Finden Sie Funktionsgleichungen von Funktionen, die die folgenden Nullstellen besitzen.
	Überprüfen Sie Ihre Lösungen anschließend mit Geogebra\footnote{\url{https://www.geogebra.com/graphing}}.
	
	\begin{enumerate}[label=\alph*)]
		\item $x_1 = 3,\; x_2 = -3$
		\item $x_1 = 3,\; x_2 = 3$
		\item $x_1 = 0,\; x_2 = 0$
		\item $x_1 = -2,\; x_2 = 0$
		\item $x_1 = -1,\; x_2 = 2,\; x_3 = 4$
	\end{enumerate}
	
	\checkered[6cm]
	
	\partnertask{Diskriminante und Nullstellen}
	
	\begin{wrapfigure}[3]{r}{4cm}
		\qrlink{https://www.geogebra.org/classic/gp2e8m5v}
	\end{wrapfigure}
	
	Beim Lösen quadratischer Gleichungen bezeichnet man den Wert unter der Wurzel in der abc- oder pq-Formel als \textit{Diskriminante} $D$.
	
	Untersuchen Sie den Zusammenhang zwischen der Anzahl der Nullstellen der Parabel und dem Wert der Diskriminante.
	Was fällt Ihnen auf?
	\vspace{2cm}
	
	\checkered[6cm]
	
	\warning{Übersicht über Darstellungsformen quadratischer Funktionen}{
		\begin{figure}[H]
			\centering
			\begin{tikzpicture}
				\node[entity, align=center] (spf) at (3.5,3){Scheitelpunktform\\$a(x-d)^2+e$};
				\node[entity, align=center] (pf) at (0,-3){Produktform\\$a(x-x_1)(x-x_2)$};
				\node[entity, align=center] (nf) at (-3.5,3){Normalform\\$ax^2+bx+c$};
				\draw[->] (nf) edge[bend right=15] node[midway, above]{Quadr. Ergänzung} (spf);
				\draw[->] (spf) edge[bend right=15] node[midway, below]{Ausmultiplizieren} (nf);
				\draw[->] (nf) edge[bend right=15] node[midway, sloped, below]{Nst. berechnen} (pf);
				\draw[->] (pf) edge[bend right=15] node[midway, sloped, above]{Ausmultiplizieren} (nf);
				\draw[->] (pf) edge[bend right=15] node[midway, sloped, below]{Scheitel berechnen} (spf);
				\draw[->] (spf) edge[bend right=15] node[midway, sloped, above]{Nst. berechnen} (pf);
			\end{tikzpicture}
		\end{figure}
	}
\end{document}