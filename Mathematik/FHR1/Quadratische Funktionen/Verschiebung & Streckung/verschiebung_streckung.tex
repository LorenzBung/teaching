\documentclass[11pt, a4paper, oneside]{article}
\usepackage[ngerman]{babel}
\usepackage{worksheet}

\begin{document}
	\author{L. Bung}
	\title{Quadratische Funktionen: \hspace{10cm} Verschiebung und Streckung}
	\subject{Mathematik}
	\class{FHR1}
	\maketitle
	
	\partnertask{Quadratische Funktionen im Alltag}
		
	\begin{wrapfigure}[3]{r}{4cm}
		\qrlink{https://www.geogebra.org/classic/fvbdwgn6}
	\end{wrapfigure}
	
	Ganz viele Alltagsgegenstände werden von quadratischen Funktionen beschrieben.
	
	a) Öffnen Sie die Geogebra-App und finden Sie die passenden Funktionsgleichungen.
	Vergleichen Sie mit Ihrem Sitznachbarn.
	\vspace{2cm}
	
	b) Finden Sie heraus, für welche Veränderung des Graphen die Koeffizienten $a$, $d$ und $e$ verantwortlich sind.
	
	\begin{table}[H]
		\centering
		\renewcommand{\arraystretch}{1.8}
		\begin{tabularx}{\textwidth}{| l | X |}
			\hline
			\textbf{Koeffizient} & \textbf{Veränderung}\\
			\hline
			\hline
			&\\
			$a$ & \\
			&\\
			\hline
			&\\
			$d$ & \\
			&\\
			\hline
			&\\
			$e$ & \\
			&\\
			\hline
		\end{tabularx}
		\renewcommand{\arraystretch}{1.5}
	\end{table}
	
	c) Überführen Sie die ersten drei Funktionsgleichungen aus Aufgabe a) in Polynome der Form $f(x) = ax^2 + bx + c$. Multiplizieren Sie dazu die Funktionsgleichungen aus.
	Sie können dafür die binomischen Formeln nutzen.
	
	\checkered[10cm]
	
	\hint{Scheitelpunktform und Normalform}{
		\phantom{x}\vspace{9cm}
	}
	
	\singletask{Von der Normalform zur Scheitelpunktform}
	
	Wie man von der Scheitelpunktform zur Normalform kommt, haben wir schon in Aufgabe 1c) gesehen.
	Doch wie ist es umgekehrt -- von der Normalform zurück in die Scheitelpunktform?
	
	Finden Sie die Scheitelpunktform der Funktion $f(x) = 3x^2 + 12x - 21$.
	Beschreiben Sie, wie Sie zu Ihrer Lösung kamen!
	
	\checkered[7cm]
	
	\hint{Quadratische Ergänzung}{
		\phantom{x}\vspace{8.5cm}
	}
\end{document}