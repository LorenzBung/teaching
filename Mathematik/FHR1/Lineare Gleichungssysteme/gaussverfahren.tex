\documentclass[11pt, a4paper, oneside]{article}
\usepackage[ngerman]{babel}
\usepackage{worksheet}

\begin{document}
	\author{L. Bung}
	\title{Lineare Gleichungssysteme: \hspace{10cm} Gaußverfahren}
	\subject{Mathematik}
	\class{FHR1}
	\maketitle
	
	\singletask{Ein Parameter}
	
	Die Gerade $g(x) = mx + 2$ geht durch den Punkt $P(3|1)$.
	Bestimmen Sie den Parameter $m$.
	
	\checkered[3cm]
	
	\singletask{Zwei Parameter}
	
	Eine Gerade $h(x) = mx + c$ geht durch die Punkte $Q(1|1)$ und $R(4|7)$.
	Bestimmen Sie die Parameter $m$ und $c$.
	
	\checkered[5cm]
	
	\partnertask{$n$ Parameter}
	
	Wie viele Punkte braucht man, um 3/4/$n$ Parameter zu bestimmen?
	
	\lines[3cm]
	
	\hint{Gaußverfahren}{
		\phantom{text}
		\vspace{7cm}
	}
	
	\singletask{LGS lösen}
	
	Geben Sie die Lösungsmenge des linearen Gleichungssystems an:
	\begin{align*}
		2x_1 + 2x_2 + 3x_3 &= 75\\
		x_1 + 2x_2 + 2x_3 &= 50\\
		x_1 + x_2 + 2x_3 &= 40\\
	\end{align*}
	
	\checkered[7cm]
\end{document}