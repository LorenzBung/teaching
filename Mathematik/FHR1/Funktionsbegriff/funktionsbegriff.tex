\documentclass[11pt, a4paper, oneside]{article}
\usepackage[ngerman]{babel}
\usepackage{worksheet}

\begin{document}
	\author{L. Bung}
	\title{Funktionsbegriff}
	\subject{Mathematik}
	\class{FHR1}
	\maketitle
	
	\hint{Funktionsbegriff}{\phantom{text}\vspace{7cm}}
	
	\singletask{Beispiele und Gegenbeispiele}
	
	Zeichnen Sie in das Koordinatensystem jeweils ein Beispiel für eine Funktion und eine Nicht-Funktion.
	
	\begin{plot}[xmin=-4, xmax=4, ymin=-4, ymax=4]
	\end{plot}
	
	\hint{Schreibweisen für Funktionen}{\phantom{text}\vspace{7cm}}
	
	\singletask{Notationswechsel}
	
	Schreiben Sie die Funktionsgleichung jeweils mithilfe der beiden anderen Schreibweisen.
	
	\begin{multicols}{3}
		\begin{enumerate}[label=\alph*)]
			\item $y = 3x+2$
			\item $g(x) = \frac{1}{2}x-2$
			\item $h: x \mapsto 2x-4$
		\end{enumerate}
	\end{multicols}
	
	\checkered[9cm]
\end{document}