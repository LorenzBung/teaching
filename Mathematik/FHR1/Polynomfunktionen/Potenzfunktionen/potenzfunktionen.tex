\documentclass[11pt, a4paper, oneside]{article}
\usepackage[ngerman]{babel}
\usepackage{worksheet}

\begin{document}
	\author{L. Bung}
	\title{Potenzfunktionen}
	\subject{Mathematik}
	\class{FHR1}
	\maketitle
	
	\begin{hint}{Potenzfunktionen}
		Funktionen mit Funktionsgleichungen der Form $f(x) = a \cdot x^n$ mit $a \in \mathbb{R}$ und $n \in \mathbb{N}$ nennt man \textbf{Potenzfunktionen vom Grad $n$}.
	\end{hint}
	
	\partnertask{Verlauf der Graphen von Potenzfunktionen}
	
	a) Füllen Sie die folgende Tabelle aus:
	
	\begin{table}[H]
		\centering
		\begin{tabularx}{\textwidth}{| X | p{1.5cm} | p{1.5cm} | p{1.5cm} | p{1.5cm} | p{1.5cm} | p{1.5cm} |}
			\hline
			$x$ & -2 & -1 & 0 & $\frac{1}{2}$ & 1 & 2\\
			\hline
			\hline
			$f(x) = \frac{1}{10} x^3$ &&&&&&\\
			\hline
			$g(x) = \frac{1}{10} x^4$ &&&&&&\\
			\hline
			$h(x) = \frac{1}{10} x^5$ &&&&&&\\
			\hline
			$i(x) = \frac{1}{10} x^6$ &&&&&&\\
			\hline
		\end{tabularx}
	\end{table}
	
	b) Zeichnen Sie die Graphen der Funktionen $f(x)$, $g(x)$, $h(x)$ und $i(x)$ in ein Koordinatensystem.
	Verwenden Sie dazu die Tabelle aus Teilaufgabe a).
	
	\checkered[8.5cm]
	
	c) Beschreiben Sie, wie die Graphen der Funktionen verlaufen.
	Welche Muster fallen Ihnen auf?
	Wie verlaufen wohl die Graphen der Funktionen $j(x) = \frac{1}{10} x^7$ und $k(x) = \frac{1}{10} x^8$?
	
	\lines[5cm]
	
	\begin{hint}{Verlauf der Graphen von Potenzfunktionen}
		\phantom{x}\vspace{8cm}
	\end{hint}
	
	\singletask{Funktionsgleichungen von Potenzfunktionen aufstellen}
	
	Bestimmen Sie die Funktionsgleichung der Potenzfunktion $f(x) = a \cdot x^n$, die durch die beiden Punkte $P$ und $Q$ geht:
	
	\begin{multicols}{2}
		\begin{enumerate}[label=\alph*)]
			\item $P(1\,|\,0,5)$ und $Q(-2\,|\,32)$
			\item $P(1\,|\,0,4)$ und $Q(4\,|\,25,6)$
		\end{enumerate}
	\end{multicols}
	
	c) Kann man \textit{ohne Rechnung} erkennen, ob die Potenzfunktion bei a) oder b) eine ungerade oder eine gerade Hochzahl hat?
	
	\checkered[13cm]
	
	\bonustask{Verschiebung und Streckung von Potenzfunktionen}
	
	Überlegen Sie sich, wie man Potenzfunktionen durch Veränderung der Funktionsgleichung...
	\begin{itemize}
		\item ...nach oben und unten verschieben kann.
		\item ...in y-Richtung strecken und stauchen kann.
		\item ...nach links und rechts verschieben kann.
	\end{itemize}
	
	Überprüfen Sie Ihre Vermutung(en) mit Geogebra\footnote{\url{https://www.geogebra.org/calculator}}.
	
	\lines[3cm]
\end{document}