\documentclass[11pt, a4paper, oneside]{article}
\usepackage[ngerman]{babel}
\usepackage{worksheet}

\begin{document}
	\author{L. Bung}
	\title{Vektorgeometrie: Skalarprodukt}
	\subject{Mathematik}
	\class{TG11}
	\maketitle
	
	\partnertask{Rechte Winkel zwischen Vektoren}
	
	In der folgenden Darstellung stehen die Vektoren $\vec{a}$ und $\vec{b}$ im rechten Winkel zueinander.
	Man sagt auch: Die Vektoren $\vec{a}$ und $\vec{b}$ sind zueinander \textbf{orthogonal} (Schreibe: $\vec{a}\perp\vec{b}$).
	
	\begin{figure}[H]
		\centering
		\begin{tikzpicture}
			\tkzDefPoint(0,0){O}
			\tkzDefPoint(1,3){A}
			\tkzDefPoint(6,-2){B}
			\tkzDrawLine[thick, -latex, add = 0 and 0](O,A)
			\tkzLabelSegment[left](O,A){$\vec{a} = \begin{pmatrix}1\\3\end{pmatrix}$}
			\tkzDrawLine[thick, -latex, add = 0 and 0](O,B)
			\tkzLabelSegment[below left](O,B){$\vec{b} = \begin{pmatrix}6\\-2\end{pmatrix}$}
			\tkzDrawLine[add = 0 and 0](A,B)
			\tkzMarkRightAngle[german, size=.5](B,O,A)
		\end{tikzpicture}
	\end{figure}

	a) Überlegen Sie sich eine Möglichkeit, wie man nachrechnen kann, dass es sich tatsächlich um einen rechten Winkel handelt.
	
	\checkered[10cm]
	
	b) Wie könnte man bei zwei beliebigen Vektoren $\vec{a} = \begin{pmatrix}a_1\\a_2\end{pmatrix}$ und $\vec{b}=\begin{pmatrix}b_1\\b_2\end{pmatrix}$ überprüfen, ob der Winkel zwischen ihnen 90° beträgt?
	
	\checkered
	
	c) Auch Vektoren im dreidimensionalen Raum können zueinander orthogonal sein:
	Zum Beispiel die Vektoren $\vec{a} = \begin{pmatrix}1\\2\\3\end{pmatrix}$ und $\vec{b} = \begin{pmatrix}6\\0\\-2\end{pmatrix}$.
	Wie kann man das Konzept von oben auf den dreidimensionalen Fall erweitern?
	
	\checkered
	
	\boxarea[9cm]
\end{document}
