\documentclass[11pt, a4paper, oneside]{article}
\usepackage[ngerman]{babel}
\usepackage{worksheet}

\begin{document}
	\author{L. Bung}
	\title{Malen mit Geogebra}
	\subject{Mathematik}
	\maketitle
	
	In Geogebra können Funktionen (wie z. B. $f(x) = \frac{1}{2}x^2$) auch nur in bestimmten Bereichen definiert werden.
	Das geht mit dem folgenden Code:
	
	\begin{lstlisting}
Funktion[<Funktionsterm>, <Startwert>, <Endwert>]
Funktion[0.5x^2, 0, 2]
	\end{lstlisting}
	
	Wir können uns das zunutze machen, um Bilder zu malen!
	
	\task{Malen mit Geogebra}
	
	Erstellen Sie ein Bild mithilfe von mehreren Funktionen in bestimmten Bereichen.
	
	\textbf{Tipps}:
	
	\begin{itemize}
		\item Sie können zusätzlich auch noch die Farbe ändern, in der die Funktionen angezeigt werden.
		\item Mit dem Geogebra-Befehl \texttt{Circle} können auch Kreise gezeichnet werden.
		\item Mit dem Flächen-Werkzeug können Flächen farbig markiert werden.
	\end{itemize}
	
	Speichern Sie Ihr fertiges Bild am Ende (oder machen Sie einen Screenshot).
	
	Viel Spaß!
\end{document}