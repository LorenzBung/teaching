\documentclass[11pt, a4paper, oneside]{article}
\usepackage{worksheet}

\begin{document}
	
	\makeheader{Mathematik}{Achsenspiegelung}
	
	\singletask*{Wiederholung: Punkte im Koordinatensystem}
	
	a) Zeichnen Sie die folgenden Punkte in ein Koordinatensystem ein: $A(1|1)$, $B(3|1)$, $C(1|3)$, $D(-1|0)$, $E(4|-1)$.
	
	\checkered[6cm]
	b) Überprüfen Sie mithilfe eines Thaleskreises, ob die Gerade $g$ durch $B$ und $D$ und die Gerade $h$ durch $B$ und $E$ senkrecht aufeinander stehen.

	\singletask{Symmetrieachsen}

	Zeichnen Sie alle Symmetrieachsen ein.
	
	\begin{checkeredfigure}
		\node[label={a)}] at (2.25,5.75){};
		\draw[very thick] (4,6) -- (3,2) -- (5,2) -- (4,6);
		
		\node[label={b)}] at (7.25,5.75){};
		\draw[very thick] (8,2) -- (10,2) -- (10,6) -- (8,6) -- (8,2);
		
		\node[label={c)}] at (12.25,5.75){};
		\draw[very thick] (13,3) -- (15,3) -- (15,5) -- (13,5) -- (13,3);
	\end{checkeredfigure}

	d) Wie viele Symmetrieachsen hat ein Kreis? \hrulefill
	
	\pagebreak
	\hint{Achsensymmetrie}{Eine Figur, die durch Spiegelung an einer Geraden auf sich selbst abgebildet wird, heißt \textbf{achsensymmetrisch}. Die Gerade nennt man \textbf{Symmetrieachse} oder \textbf{Spiegelachse}.}
	
	\singletask{Dreiecke spiegeln}
	
	a) Spiegeln Sie das Dreieck $\Delta ABC$ an der vorgegebenen Achse.

	\begin{checkeredfigure}
		\tkzDefPoint(3,2){A}
		\tkzDefPoint(5,4){B}
		\tkzDefPoint(2,6){C}
		\tkzDrawPoints(A,B,C)
		\tkzLabelPoints(A,B)
		\tkzLabelPoint[above](C){$C$}
		\tkzDrawSegment[thick](A,B)
		\tkzDrawSegment[thick](B,C)
		\tkzDrawSegment[thick](C,A)
		
		\draw[ultra thick] (8,1) -- (8,7);
	\end{checkeredfigure}

	b) Spiegeln Sie das Dreieck $\Delta DEF$ an der vorgegebenen Achse.
	
	\begin{checkeredfigure}
		\tkzDefPoint(5,2){D}
		\tkzDefPoint(11,3){E}
		\tkzDefPoint(8,7){F}
		\tkzDrawPoints(D,E,F)
		\tkzLabelPoints(D,E)
		\tkzLabelPoint[left](F){$F$}
		\draw[thick] (D) -- (E) -- (F) -- (D);
		\draw[ultra thick] (2,4) -- (14,4);
	\end{checkeredfigure}
	
\end{document}
