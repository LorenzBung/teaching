\documentclass[11pt, a4paper, oneside]{article}
\usepackage{worksheet}

\begin{document}
	
	\makeheader[2BFE1]{Mathematik}{Strahlensätze}
	
	\partnertask{Längen und Verhältnisse}
	
	\begin{figure}[h]
		\centering
		\begin{minipage}{.3\textwidth}
		\centering
		\begin{tikzpicture}
			\tkzDefPoint(0,0){S}
			\tkzDefPoint(3,0){A}
			\tkzDefPoint(3,2){B}
			\tkzDefPoint(4,0){C}
			\tkzDefPoint(4,2.666){D}
			\tkzDrawPoints(S,A,B,C,D)
			\tkzLabelPoints(S,A,C)
			\tkzLabelPoint[above left](B){$B$}
			\tkzLabelPoint[above left](D){$D$}
			
			\tkzDrawLine[add = .5cm and .5cm](S,C)
			\tkzDrawLine[add = .5cm and .5cm](S,D)
			\tkzDrawSegment(A,B)
			\tkzDrawSegment(C,D)
		\end{tikzpicture}
		\end{minipage}
		\hfill
		\begin{minipage}{.3\textwidth}
		\centering
		\begin{tikzpicture}
			\tkzDefPoint(0,0){S}
			\tkzDefPoint(3,0){A}
			\tkzDefPoint(2.3,2.3){B}
			\tkzDefPoint(4,0){C}
			\tkzDefPoint(3,3){D}
			\tkzDrawPoints(S,A,B,C,D)
			\tkzLabelPoints(S,A,C)
			\tkzLabelPoint[above left](B){$B$}
			\tkzLabelPoint[above left](D){$D$}
			
			\tkzDrawLine[add = .5cm and .5cm](S,C)
			\tkzDrawLine[add = .5cm and .5cm](S,D)
			\tkzDrawSegment(A,B)
			\tkzDrawSegment(C,D)
		\end{tikzpicture}
		\end{minipage}
		\hfill
		\begin{minipage}{.3\textwidth}
		\centering
		\begin{tikzpicture}
			\tkzDefPoint(0,0){S}
			\tkzDefPoint(2,0){A}
			\tkzDefPoint(3,2){B}
			\tkzDefPoint(4,0){C}
			\tkzDefPoint(4,2.666){D}
			\tkzDrawPoints(S,A,B,C,D)
			\tkzLabelPoints(S,A,C)
			\tkzLabelPoint[above left](B){$B$}
			\tkzLabelPoint[above left](D){$D$}
			
			\tkzDrawLine[add = .5cm and .5cm](S,C)
			\tkzDrawLine[add = .5cm and .5cm](S,D)
			\tkzDrawSegment(A,B)
			\tkzDrawSegment(C,D)
		\end{tikzpicture}
		\end{minipage}
	\end{figure}

	Füllen Sie die Tabelle aus.
	Messen Sie dazu die Längen und berechnen Sie die Verhältnisse.
	
	{
	\renewcommand{\arraystretch}{2}
	\setlength\tabcolsep{.5cm}
	\begin{tabularx}{\textwidth}{c|X|X|X}
		& Linke Abbildung & Mittlere Abbildung & Rechte Abbildung\\
		\hline
		$\overline{SA}$ &&&\\
		\hline
		$\overline{SC}$ &&&\\
		\hline
		$\overline{SB}$ &&&\\
		\hline
		$\overline{SD}$ &&&\\
		\hline
		$\dfrac{\ \overline{SA}\ }{\overline{SC}}$ &&&\\
		\hline
		$\dfrac{\ \overline{SB}\ }{\overline{SD}}$ &&&\\
	\end{tabularx}
	}

	Was fällt Ihnen auf, wenn Sie die Verhältnisse miteinander vergleichen?
	
	\lines[3cm]
	
	\pagebreak
	
	\boxarea[22cm]
	
\end{document}
