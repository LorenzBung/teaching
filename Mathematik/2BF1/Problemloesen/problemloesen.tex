\documentclass[11pt, a4paper, oneside]{article}
\usepackage[ngerman]{babel}
\usepackage{worksheet}
\usepackage{pdflscape}

\begin{document}
	\author{L. Bung}
	\title{Problemlösen}
	\subject{Mathematik}
	\class{2BF1}
	\maketitle
	
	\grouptask[{\faClock[regular]} 30 min.]{Treppen unterschiedlicher Länge}
	
	Die Firma Müller stellt unterschiedlich lange Treppen her.
	Jede Treppenstufe hat die Form eines Würfels mit der Kantenlänge 30 cm.

	\begin{figure}[H]
		\begin{subfigure}[b]{.3\textwidth}
			\centering
			\tdplotsetmaincoords{70}{80}
			\begin{tikzpicture}[tdplot_main_coords]
				\draw (0,0,0) -- (0,0,1);
				\draw (0,0,0) -- (1,0,0);
				\draw (1,0,0) -- (1,0,1);
				\draw (0,0,1) -- (1,0,1);
				\draw (1,0,0) -- (1,1,0);
				\draw (1,0,1) -- (1,1,1);
				\draw (1,1,0) -- (1,1,1);
				\draw (0,0,1) -- (0,1,1);
				\draw (1,1,1) -- (0,1,1);
			\end{tikzpicture}
			\caption{Treppe mit $n = 1$ Stufen}
		\end{subfigure}
		\begin{subfigure}[b]{.3\textwidth}
			\centering
			\tdplotsetmaincoords{70}{80}
			\begin{tikzpicture}[tdplot_main_coords]
				\draw (0,0,0) -- (0,0,1);
				\draw (0,0,0) -- (1,0,0);
				\draw (1,0,0) -- (1,0,1);
				\draw (0,0,1) -- (1,0,1);
				\draw (1,0,0) -- (1,1,0);
				\draw (1,0,1) -- (1,1,1);
				\draw (1,1,0) -- (1,1,1);
				\draw (0,0,1) -- (0,1,1);
				\draw (1,1,1) -- (0,1,1);
				
				\draw (1,1,0) -- (1,2,0);
				\draw (1,1,1) -- (1,2,1);
				\draw (1,2,0) -- (1,2,1);
				\draw (0,1,1) -- (0,1,2);
				\draw (1,1,1) -- (1,1,2);
				\draw (1,2,1) -- (1,2,2);
				\draw (0,1,2) -- (0,2,2);
				\draw (0,1,2) -- (1,1,2);
				\draw (1,1,2) -- (1,2,2);
				\draw (0,2,2) -- (1,2,2);
			\end{tikzpicture}
			\caption{Treppe mit $n = 2$ Stufen}
		\end{subfigure}
		\begin{subfigure}[b]{.3\textwidth}
			\centering
			\tdplotsetmaincoords{70}{80}
			\begin{tikzpicture}[tdplot_main_coords]
				\draw (0,0,0) -- (0,0,1);
				\draw (0,0,0) -- (1,0,0);
				\draw (1,0,0) -- (1,0,1);
				\draw (0,0,1) -- (1,0,1);
				\draw (1,0,0) -- (1,1,0);
				\draw (1,0,1) -- (1,1,1);
				\draw (1,1,0) -- (1,1,1);
				\draw (0,0,1) -- (0,1,1);
				\draw (1,1,1) -- (0,1,1);
				
				\draw (1,1,0) -- (1,2,0);
				\draw (1,1,1) -- (1,2,1);
				\draw (1,2,0) -- (1,2,1);
				\draw (0,1,1) -- (0,1,2);
				\draw (1,1,1) -- (1,1,2);
				\draw (1,2,1) -- (1,2,2);
				\draw (0,1,2) -- (0,2,2);
				\draw (0,1,2) -- (1,1,2);
				\draw (1,1,2) -- (1,2,2);
				\draw (0,2,2) -- (1,2,2);
				
				\draw (1,2,0) -- (1,3,0);
				\draw (1,2,1) -- (1,3,1);
				\draw (1,2,2) -- (1,3,2);
				\draw (1,3,0) -- (1,3,1);
				\draw (1,3,1) -- (1,3,2);
				\draw (0,2,2) -- (0,2,3);
				\draw (1,2,2) -- (1,2,3);
				\draw (1,3,2) -- (1,3,3);
				\draw (0,2,3) -- (1,2,3);
				\draw (0,2,3) -- (0,3,3);
				\draw (0,3,3) -- (1,3,3);
				\draw (1,2,3) -- (1,3,3);
			\end{tikzpicture}
			\caption{Treppe mit $n = 3$ Stufen}
		\end{subfigure}
	\end{figure}
	
	\begin{enumerate}[label=\alph*)]
		\item Die Treppe besteht aus Beton.
		Wie viel Beton wird benötigt, um eine Treppe mit $n$ Stufen zu bauen?
		(Tipp: Bei 30 cm Kantenlänge hat jeder Würfel ein Volumen von 27 000 $\mathrm{cm}^3$.)
		\item Die Treppe wird rundherum mit Holz verkleidet.
		Je nachdem, wie viele Stufen die Treppe hat, wird unterschiedlich viel Holz benötigt.
		Wie viel Holz wird gebraucht, wenn die Treppe $n$ Stufen hat?
		(Tipp: Bei 30 cm Kantenlänge hat jede Stufe eine Oberfläche von 900 $\mathrm{cm}^2$.)
	\end{enumerate}
	
	\begin{itemize}
		\item Wählen Sie entweder Aufgabe a (leichter) oder b (schwerer) aus -- je nachdem, was Sie sich als Gruppe zutrauen.
		\item Versuchen Sie, das Problem gemeinsam in der Gruppe zu lösen.
		\item Überlegen Sie sich zusammen Strategien, wie man an die Aufgabe herangehen kann.
		\item Schreiben Sie auf, welche der Strategien sie verwenden und ob sie hilfreich war.
		\item Wenn Sie nicht weiterkommen, können Sie sich die Hilfekarten anschauen, die im Raum verteilt liegen.
	\end{itemize}

	\grouptask[{\faClock[regular]} 10 min.]{Präsentation}
	
	Präsentieren Sie als Gruppe Ihre Lösungsstrategie(n).
	Welche Strategie war hilfreich?
	An welchen Stellen sind Sie mit der Strategie nicht weitergekommen?
	
	\pagebreak
	
	\begin{landscape}
		\pagenumbering{gobble}
		\begin{center}
			\fontsize{30pt}{10pt}\selectfont
			\vspace*{2cm}
			\textbf{Strategie: Aufteilen}
		\end{center}
		
		\vspace{2cm}
		\huge
		\begin{itemize}
			\item Kann man die Situation in kleine Einzelprobleme zerlegen?
			\item Was passiert, wenn man eines der Einzelprobleme verändert?
			\item Wie verändert sich dann das Gesamtproblem?
			\item Kann ich die Einzelprobleme einzeln lösen und damit auch die Gesamtaufgabe lösen?
		\end{itemize}
	\end{landscape}

	\pagebreak
	
	\begin{landscape}
		\begin{center}
			\fontsize{30pt}{10pt}\selectfont
			\vspace*{2cm}
			\textbf{Strategie: Darstellung ändern}
		\end{center}
	
		\vspace{2cm}
		\huge
		Kann man die Situation...
		\begin{itemize}
			\item als Bild
			\item in einer Tabelle
			\item durch Zahlen oder Formeln
		\end{itemize}
		... darstellen? Hilft die andere Darstellung vielleicht weiter?
	\end{landscape}

	\begin{landscape}
		\begin{center}
			\fontsize{30pt}{10pt}\selectfont
			\vspace*{2cm}
			\textbf{Strategie: Muster suchen}
		\end{center}
		
		\vspace{2cm}
		\huge
		\begin{itemize}
			\item Welche Eigenschaften wiederholen sich oder sind regelmäßig?
			\item Warum sind die Muster so?
			\item Können sie auch anders sein? Warum / warum nicht?
		\end{itemize}
	\end{landscape}

	\begin{landscape}
		\begin{center}
			\fontsize{30pt}{10pt}\selectfont
			\vspace*{2cm}
			\textbf{Strategie: Ausprobieren}
		\end{center}
		
		\vspace{2cm}
		\huge
		\begin{itemize}
			\item Kann man einige (oder alle) Fälle ausprobieren?
			\item Wann kommt man durch probieren nicht weiter?
			\item Warum geht es in diesen Fällen nicht?
		\end{itemize}
	\end{landscape}

	\begin{landscape}
		\begin{center}
			\fontsize{30pt}{10pt}\selectfont
			\vspace*{2cm}
			\textbf{Strategie: Aufgabe verändern}
		\end{center}
		
		\vspace{2cm}
		\huge
		\begin{itemize}
			\item Kann man die Aufgabe lösen, wenn man sie vereinfacht?
			\item Wie kommt man vom vereinfachten Fall zurück zur ursprünglichen Aufgabe?
			\item Kann man die Vorgehensweise vom vereinfachten Fall so erweitern, dass die ursprüngliche Aufgabe damit gelöst wird?
		\end{itemize}
	\end{landscape}

	\pagebreak
	\pagenumbering{arabic}
	\author{L. Bung}
	\title{Merkblatt\hspace{10cm}Problemlösen}
	\subject{Mathematik}
	\class{2BF1}
	\maketitle
	
	Bei schwierigen Aufgaben lassen sich häufig Strategien zur Lösung anwenden.
	
	\tdplotsetmaincoords{70}{80}
	\renewcommand{\arraystretch}{2}
	\setlength\tabcolsep{.5cm}
	\begin{tabularx}{\textwidth}{l X X}
		\hline
		\textbf{Strategie} & \textbf{Beschreibung} & \textbf{Beispiel}\\
		\hline
		Aufteilen & Aufgabe in kleine Probleme zerlegen, die einfacher zu lösen sind. &
		\vspace{-.25cm}
		\begin{tikzpicture}[tdplot_main_coords, baseline=10pt]
			\draw (0,0,0) -- (0,-.5,1);
			\draw (0,0,0) -- (0,.5,1);
			\tdplotdrawarc{(0,0,1)}{.5}{0}{360}{}{};
			\tdplotdefinepoints(0,0,1)(0,-.5,1)(0,0,1.5);
			\tdplotdrawpolytopearc{.5}{}{};
			\tdplotdefinepoints(0,0,1)(0,0,1.5)(0,.5,1);
			\tdplotdrawpolytopearc{.5}{}{};
		\end{tikzpicture} =
		\begin{tikzpicture}[tdplot_main_coords, baseline=10pt]
			\draw (0,0,0) -- (0,-.5,1);
			\draw (0,0,0) -- (0,.5,1);
			\tdplotdrawarc{(0,0,1)}{.5}{0}{360}{}{};
		\end{tikzpicture} +
		\begin{tikzpicture}[tdplot_main_coords]
			\tdplotdrawarc{(0,0,1)}{.5}{0}{360}{}{};
			\tdplotdefinepoints(0,0,1)(0,-.5,1)(0,0,1.5);
			\tdplotdrawpolytopearc{.5}{}{};
			\tdplotdefinepoints(0,0,1)(0,0,1.5)(0,.5,1);
			\tdplotdrawpolytopearc{.5}{}{};
		\end{tikzpicture}\\
	
		Darstellung ändern & Aufgabe anders darstellen: Als Bild, als Tabelle, als Formel... &
		\vspace{-.5cm}
		\begin{tikzpicture}[tdplot_main_coords, baseline=2pt]
			\draw (0,0,0) -- (0,0,.5);
			\draw (0,0,0) -- (.5,0,0);
			\draw (.5,0,0) -- (.5,0,.5);
			\draw (0,0,.5) -- (.5,0,.5);
			\draw (.5,0,0) -- (.5,.5,0);
			\draw (.5,0,.5) -- (.5,.5,.5);
			\draw (.5,.5,0) -- (.5,.5,.5);
			\draw (0,0,.5) -- (0,.5,.5);
			\draw (.5,.5,.5) -- (0,.5,.5);
			\node[below] at (.5,.25,0) {$a$};
			\node[right] at (.5,.5,.25) {$a$};
			\node[below left] at (.25,0,0) {$a$};
		\end{tikzpicture}\hspace{.5cm} $\Rightarrow V = a^3$\\
	
		Muster suchen & Regelmäßigkeiten finden und testen, wann sie nicht mehr gelten. &
		\vspace{-.25cm}
		\begin{tikzpicture}[tdplot_main_coords, baseline=5pt]
			\draw (0,0,0) -- (0,0,.5);
			\draw (0,0,0) -- (.5,0,0);
			\draw (.5,0,0) -- (.5,0,.5);
			\draw (0,0,.5) -- (.5,0,.5);
			\draw (.5,0,0) -- (.5,.5,0);
			\draw (.5,0,.5) -- (.5,.5,.5);
			\draw (.5,.5,0) -- (.5,.5,.5);
			\draw (0,0,.5) -- (0,.5,.5);
			\draw (.5,.5,.5) -- (0,.5,.5);
			
			\draw (.5,.5,0) -- (.5,1,0);
			\draw (.5,.5,.5) -- (.5,1,.5);
			\draw (.5,1,0) -- (.5,1,.5);
			\draw (0,.5,.5) -- (0,.5,1);
			\draw (.5,.5,.5) -- (.5,.5,1);
			\draw (.5,1,.5) -- (.5,1,1);
			\draw (0,.5,1) -- (0,1,1);
			\draw (0,.5,1) -- (.5,.5,1);
			\draw (.5,.5,1) -- (.5,1,1);
			\draw (0,1,1) -- (.5,1,1);
			
			\node[below] at (.5,.5,0) {$2$};
			\node[right] at (.5,1,.5) {$2$};
		\end{tikzpicture} =
		\begin{tikzpicture}[tdplot_main_coords, baseline=5pt]
			\draw (0,0,0) -- (0,0,.5);
			\draw (0,0,0) -- (.5,0,0);
			\draw (.5,0,0) -- (.5,0,.5);
			\draw (0,0,.5) -- (.5,0,.5);
			\draw (.5,0,0) -- (.5,.5,0);
			\draw (.5,0,.5) -- (.5,.5,.5);
			\draw (.5,.5,0) -- (.5,.5,.5);
			\draw (0,0,.5) -- (0,.5,.5);
			\draw (.5,.5,.5) -- (0,.5,.5);
			
			\node[below] at (.5,.25,0) {$1$};
			\node[right] at (.5,.5,.25) {$1$};
		\end{tikzpicture} +
		\begin{tikzpicture}[tdplot_main_coords, baseline=5pt]
			\draw (.5,.5,0) -- (.5,.5,.5);
			\draw (0,.5,0) -- (.5,.5,0);
			\draw (0,.5,0) -- (0,.5,.5);
			\draw (0,.5,.5) -- (.5,.5,.5);
			\draw (.5,.5,0) -- (.5,1,0);
			\draw (.5,.5,.5) -- (.5,1,.5);
			\draw (.5,1,0) -- (.5,1,.5);
			\draw (0,.5,.5) -- (0,.5,1);
			\draw (.5,.5,.5) -- (.5,.5,1);
			\draw (.5,1,.5) -- (.5,1,1);
			\draw (0,.5,1) -- (0,1,1);
			\draw (0,.5,1) -- (.5,.5,1);
			\draw (.5,.5,1) -- (.5,1,1);
			\draw (0,1,1) -- (.5,1,1);
			
			\node[below] at (.5,.75,0) {$1$};
			\node[right] at (.5,1,.5) {$2$};
		\end{tikzpicture}\\
	
		Ausprobieren & Mehrere Fälle ausprobieren und so herausfinden, was funktioniert und was nicht. &
		\vspace{-.5cm}
		\begin{tikzpicture}[tdplot_main_coords, baseline=2pt]
			\draw (0,0,0) -- (0,0,.5);
			\draw (0,0,0) -- (.5,0,0);
			\draw (.5,0,0) -- (.5,0,.5);
			\draw (0,0,.5) -- (.5,0,.5);
			\draw (.5,0,0) -- (.5,.5,0);
			\draw (.5,0,.5) -- (.5,.5,.5);
			\draw (.5,.5,0) -- (.5,.5,.5);
			\draw (0,0,.5) -- (0,.5,.5);
			\draw (.5,.5,.5) -- (0,.5,.5);
		\end{tikzpicture}\hspace{.5cm} $\Rightarrow V = 27\ 000\ \mathrm{cm}^3$
		\begin{tikzpicture}[tdplot_main_coords, baseline=5pt]
			\draw (0,0,0) -- (0,0,.5);
			\draw (0,0,0) -- (.5,0,0);
			\draw (.5,0,0) -- (.5,0,.5);
			\draw (0,0,.5) -- (.5,0,.5);
			\draw (.5,0,0) -- (.5,.5,0);
			\draw (.5,0,.5) -- (.5,.5,.5);
			\draw (.5,.5,0) -- (.5,.5,.5);
			\draw (0,0,.5) -- (0,.5,.5);
			\draw (.5,.5,.5) -- (0,.5,.5);
			
			\draw (.5,.5,0) -- (.5,1,0);
			\draw (.5,.5,.5) -- (.5,1,.5);
			\draw (.5,1,0) -- (.5,1,.5);
			\draw (0,.5,.5) -- (0,.5,1);
			\draw (.5,.5,.5) -- (.5,.5,1);
			\draw (.5,1,.5) -- (.5,1,1);
			\draw (0,.5,1) -- (0,1,1);
			\draw (0,.5,1) -- (.5,.5,1);
			\draw (.5,.5,1) -- (.5,1,1);
			\draw (0,1,1) -- (.5,1,1);
		\end{tikzpicture}\rule{0pt}{1.5cm} \hspace{.1cm} $\Rightarrow V = 3 \cdot 27\ 000\ \mathrm{cm}^3$\\
	
		Aufgabe verändern & Aufgabe leichter machen, um sie lösen zu können. Anschließend von der Lösung auf die Lösung der ursprünglichen Aufgabe schließen. &
		\vspace{-.5cm}
		\begin{tikzpicture}[tdplot_main_coords, baseline=30pt]
			\draw (0,0,0) -- (0,0,.5);
			\draw (0,0,0) -- (.5,0,0);
			\draw (.5,0,0) -- (.5,0,.5);
			\draw (0,0,.5) -- (.5,0,.5);
			\draw (.5,0,0) -- (.5,.5,0);
			\draw (.5,0,.5) -- (.5,.5,.5);
			\draw (.5,.5,0) -- (.5,.5,.5);
			\draw (0,0,.5) -- (0,.5,.5);
			\draw (.5,.5,.5) -- (0,.5,.5);
			
			\node[below] at (.5,.25,0) {$n=1$};
		\end{tikzpicture}
		\begin{tikzpicture}[tdplot_main_coords, baseline=30pt]
			\draw (0,0,0) -- (0,0,.5);
			\draw (0,0,0) -- (.5,0,0);
			\draw (.5,0,0) -- (.5,0,.5);
			\draw (0,0,.5) -- (.5,0,.5);
			\draw (.5,0,0) -- (.5,.5,0);
			\draw (.5,0,.5) -- (.5,.5,.5);
			\draw (.5,.5,0) -- (.5,.5,.5);
			\draw (0,0,.5) -- (0,0,1);
			\draw (.5,0,.5) -- (.5,0,1);
			\draw (.5,.5,.5) -- (.5,.5,1);
			\draw (0,0,1) -- (0,.5,1);
			\draw (0,.5,1) -- (.5,.5,1);
			\draw (.5,0,1) -- (.5,.5,1);
			\draw (0,0,1) -- (.5,0,1);
			
			\node[below] at (.5,.25,0) {$n=2$};
		\end{tikzpicture}
		\begin{tikzpicture}[tdplot_main_coords, baseline=30pt]
			\draw (0,0,0) -- (0,0,.5);
			\draw (0,0,0) -- (.5,0,0);
			\draw (.5,0,0) -- (.5,0,.5);
			\draw (0,0,.5) -- (.5,0,.5);
			\draw (.5,0,0) -- (.5,.5,0);
			\draw (.5,0,.5) -- (.5,.5,.5);
			\draw (.5,.5,0) -- (.5,.5,.5);
			\draw (0,0,.5) -- (0,0,1);
			\draw (.5,0,.5) -- (.5,0,1);
			\draw (.5,.5,.5) -- (.5,.5,1);
			\draw (.5,0,1) -- (.5,.5,1);
			\draw (0,0,1) -- (.5,0,1);
			\draw (0,0,1) -- (0,0,1.5);
			\draw (.5,0,1) -- (.5,0,1.5);
			\draw (.5,.5,1) -- (.5,.5,1.5);
			\draw (0,0,1.5) -- (0,.5,1.5);
			\draw (0,0,1.5) -- (.5,0,1.5);
			\draw (.5,0,1.5) -- (.5,.5,1.5);
			\draw (0,.5,1.5) -- (.5,.5,1.5);
			
			\node[below] at (.5,.25,0) {$n=3$};
		\end{tikzpicture}\\
		\hline
	\end{tabularx}
\end{document}