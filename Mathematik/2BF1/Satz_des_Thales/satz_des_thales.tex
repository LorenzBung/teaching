\documentclass[11pt, a4paper, oneside]{article}
\usepackage{worksheet}

\begin{document}
	\author{L. Bung}
	\title{Satz des Thales}
	\subject{Mathematik}
	\class{2BF1}
	\maketitle
	
	\singletask{Winkel messen}
	
	Messen Sie die Winkel im Dreieck.
	
	\begin{figure}[h]
		\centering
		\begin{tikzpicture}
			\draw (0,0) -- (6,0);
			\draw (6,0) -- (6,3);
			\draw (6,3) -- (0,0);
			\node[label=below left:$A$] at (0,0){};
			\node[label=below right:$B$] at (6,0){};
			\node[label=above:$C$] at (6,3){};
		\end{tikzpicture}
	\end{figure}
	
	\singletask{Winkel bei Kreispunkten}
	
	Wählen Sie einen Punkt $C$ auf dem Kreisbogen und konstruieren Sie ein Dreieck $\Delta ABC$.
	Wie groß ist der Winkel am Punkt C?
	
	\begin{figure}[h]
		\centering
		\begin{tikzpicture}
			\draw (4,2) -- (4,1.9);
			\node[label=below:$A$] (A) at (4,2) {};
			\draw (12,2) -- (12,1.9);
			\node[label=below:$B$] (B) at (12,2) {};
			\draw (3,2) -- (13,2);
			\begin{scope}
				\clip (A) rectangle (12,7);
				\draw (8,2) circle (4);
			\end{scope}
			\draw (8,1.9) -- (8,2.1);
			\node[label=below:$M$] at (8,2) {};
		\end{tikzpicture}
	\end{figure}

	Wiederholen Sie das Vorgehen mit zwei weiteren Punkten $D$ und $E$ und messen Sie jeweils die Winkel.
	Was fällt Ihnen auf?

	\lines[3cm]
	
\end{document}
