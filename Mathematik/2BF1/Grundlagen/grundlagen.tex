\documentclass[11pt, a4paper, oneside]{article}
\usepackage[ngerman]{babel}
\usepackage{worksheet}

\begin{document}
	\author{L. Bung}
	\title{Wiederholung \hspace{5cm} Grundlagen Hauptschule}
	\subject{Mathematik}
	\class{2BF1}
	\maketitle
	
	\singletask[(\faBook\ S. 304--305)]{Rechnen mit ganzen Zahlen}
	
	Berechnen Sie.
	
	\begin{multicols}{3}
		\begin{enumerate}[label=\alph*)]
			\item $15 + 23$
			\item $64 - 18$
			\item $8 \cdot 17$
			\item $64 : 16$
			\item $12 + (-5)$
			\item $35 - (-14)$
			\item $-8 \cdot 5$
			\item $-12 \cdot (-4)$
			\item $-72 : (-8)$
		\end{enumerate}
	\end{multicols}
	
	\checkered[5cm]
	
	\singletask[(\faBook\ S. 305--306)]{Rechnen mit Brüchen}
	
	Berechnen Sie.
	
	\begin{multicols}{3}
		\begin{enumerate}[label=\alph*)]
			\item $\frac{2}{3} + \frac{2}{3}$
			\item $\frac{2}{3} + \frac{1}{4}$
			\item $\frac{4}{3} - \frac{2}{3}$
			\item $\frac{1}{4} - \frac{1}{2}$
			\item $\frac{1}{2} \cdot \frac{2}{3}$
			\item $\frac{3}{4} \cdot \left(-\frac{1}{2}\right)$
			\item $\frac{3}{4} : \frac{1}{4}$
			\item $\left(-\frac{1}{3}\right) : \frac{2}{3}$
			\item $\frac{\frac{3}{4}}{\frac{2}{4}}$
		\end{enumerate}
	\end{multicols}
	
	\checkered[5cm]
	
	\singletask[(\faBook\ S. 306--307)]{Rechnen mit Dezimalzahlen}
	
	Berechnen Sie. Runden Sie auf eine Nachkommastelle.
	
	\begin{multicols}{4}
		\begin{enumerate}[label=\alph*)]
			\item $5,57 + 22,01$
			\item $-4,8 - 6,01$
			\item $-4,22 \cdot 1,5$
			\item $-2,5 : 0,5$
		\end{enumerate}
	\end{multicols}
	
	\checkered[6.5cm]
	
	\singletask{Rechnen mit Brüchen und Dezimalzahlen}
	
	Berechnen Sie und geben Sie als Bruch an.
	
	\begin{multicols}{4}
		\begin{enumerate}[label=\alph*)]
			\item $\frac{1}{4} + 0,25$
			\item $-1,25 - \frac{3}{4}$
			\item $3,6 \cdot \frac{2}{9}$
			\item $\frac{2}{3} : 0,5$
		\end{enumerate}
	\end{multicols}
	
	\checkered[6.5cm]
	
	\singletask[(\faBook\ S. 308--309)]{Rechenregeln und Reihenfolge beim Rechnen}
	
	Berechnen Sie.
	
	\begin{multicols}{4}
		\begin{enumerate}[label=\alph*)]
			\item $12 : (7 - 5)$
			\item $18 \cdot -(5 + 2 \cdot 4)$
			\item $-(3 \cdot 27) : (12 \cdot \frac{3}{4})$
			\item $4 + (3 - 4 \cdot \frac{3}{2}) \cdot 2$
		\end{enumerate}
	\end{multicols}
	
	\checkered[8cm]
	
	\singletask[(\faBook\ S. 309--310)]{Potenzen und Wurzeln}
	
	Ergänzen Sie die Lücken.
	
	\begin{multicols}{4}
		\begin{enumerate}[label=\alph*)]
			\item $3^2 = \square$
			\item $2^3 = \square$
			\item $3^\square = 27$
			\item $\square^2 = 81$
			\item $\sqrt{16} = \square$
			\item $\sqrt[3]{64} = \square$
			\item $\sqrt{\square} = 49$
			\item $\sqrt[\square]{125} = 5$
		\end{enumerate}
	\end{multicols}
	
	\checkered[6cm]
	
	\singletask[(\faBook\ S. 312)]{Dreisatz}
	
	Berechnen Sie mit dem Dreisatz.
	
	\begin{enumerate}[label=\alph*)]
		\item Jakob kauft 3 Orangen zum Preis von 4,50€. Wie viel kostet eine Orange?
		\item Annas Mobilfunkvertrag kostet 8€ für 1000 MB Datenvolumen. Wie viel muss sie für 250 MB bezahlen?
		\item Ein Arbeiter braucht 12 Stunden, um ein Loch auszuheben. Wie lange brauchen drei Arbeiter dafür?
	\end{enumerate}
	
	\checkered[7cm]
	
	\singletask[(\faBook\ S. 313--314)]{Prozent- und Zinsrechnung}
	
	Berechnen Sie. Geben Sie zusätzlich an, welche Werte Prozentsatz, Prozentwert und Grundwert sind.
	
	\begin{enumerate}[label=\alph*)]
		\item Jana bekommt eine Gehaltserhöhung: Sie hat bisher 2000€ verdient und bekommt jetzt 5\% mehr. Wie viel verdient sie jetzt?
		\item Ein Liter Apfelsaft kostet statt 1,50€ nur noch 1,00 €. Um wie viel Prozent wurde der Preis reduziert?
		\item Nach einer Preiserhöhung um 50\% kostet ein Auto 30.000€. Wie viel hat es davor gekostet?
	\end{enumerate}
	
	\checkered[8.5cm]
	
	\singletask[(\faBook\ S. 314)]{Häufigkeiten und Arithmetisches Mittel}
	
	Ein Würfel wird 10 mal geworfen.
	
	Vervollständigen Sie die Tabelle und berechnen Sie den Mittelwert.
	
	\begin{table}[H]
		\centering
		\begin{tabular}{| l | p{1cm} | p{1cm} | p{1cm} | p{1cm} | p{1cm} | p{1cm} |}
			\hline
			\textbf{Zahl} & 1 & 2 & 3 & 4 & 5 & 6\\
			\hline
			\textbf{absolute Häufigkeit} & & & 1 & 3 & 1 & \\
			\hline
			\textbf{relative Häufigkeit} & $\frac{1}{10}$ & 20 \% & & & & $\frac{2}{10}$\\
			\hline
		\end{tabular}
	\end{table}
	
	\checkered[7cm]
	
	\singletask[(\faBook\ S. 311)]{Proportionale und antiproportionale Zuordnungen}
	
	Geben Sie an, ob die Zuordnung proportional, antiproportional oder keins von beidem ist.
	Begründen Sie Ihre Antwort.
	
	\begin{multicols}{2}
		\begin{enumerate}[label=\alph*)]
			\item Anzahl gekaufter Stifte $\mapsto$ Preis
			\item Alter $\mapsto$ Schuhgröße
			\item Anzahl der Arbeiter $\mapsto$ Arbeitsdauer
			\item Fahrstrecke $\mapsto$ Fahrzeit
		\end{enumerate}
	\end{multicols}
	
	\checkered[6cm]
	
	\singletask[(\faBook\ S. 310)]{Arbeiten im Koordinatensystem}
	
	Ergänzen Sie die Koordinaten der Punkte $A$, $B$ und $C$ und tragen Sie die Punkte $D(1|1)$, $E(-2|3)$ und $F(0|2)$ ein.
	
	\begin{plot}[xmin=-3.5, xmax=3.5, ymin=-3.5, ymax=3.5]
		\coordinate (A) at (1,2);
		\coordinate (B) at (-1,-1);
		\coordinate (C) at (3,-2);
		\tkzDrawPoint(A)
		\tkzDrawPoint(B)
		\tkzDrawPoint(C)
		\tkzLabelPoint[above right](A){$A(\qquad|\qquad)$}
		\tkzLabelPoint[below left](B){$B(\qquad|\qquad)$}
		\tkzLabelPoint[below left](C){$C(\qquad|\qquad)$}
	\end{plot}
	
	\singletask[(\faBook\ S. 315--317)]{Längen-, Flächen- und Raummaße}
	
	Wandeln Sie in die angegebene Einheit um:
	
	\begin{multicols}{3}
		\begin{enumerate}[label=\alph*)]
			\item 3 m in cm
			\item $3.000.000$ dm in km
			\item 5 dm in mm
			\item $3.000$ cm$^2$ in dm$^2$
			\item $0,001$ km$^2$ in m$^2$
			\item $30.000$ mm$^2$ in m$^2$
			\item $0,0003$ dm$^3$ in cm$^3$
			\item $1.500$ l in m$^3$
			\item 1 km$^3$ in l
		\end{enumerate}
	\end{multicols}
	
	\checkered[7.5cm]
	
	\singletask[(\faBook\ S. 318)]{Zeit- und Gewichtsmaße}
	
	Wandeln Sie in die angegebene Einheit um:
	
	\begin{multicols}{2}
		\begin{enumerate}[label=\alph*)]
			\item 3 h in min
			\item 72 h in d
			\item 1 d in s
			\item 30 mg in g
			\item 731,2 kg in t
			\item 2,04 t in g
		\end{enumerate}
	\end{multicols}
	
	\checkered[4.5cm]
	
	\singletask[(\faBook\ S. 318 -- 319)]{Winkel}
	
	Geben Sie für jeden der Winkel $\alpha$, $\beta$ und $\gamma$ an, ob es sich um einen stumpfen, spitzen oder rechten Winkel handelt.
	Messen Sie anschließend nach und schreiben Sie den genauen Winkel dazu.
	
	\begin{figure}[H]
		\centering
		\begin{tikzpicture}
			\tkzDefPoint(0,0){A}
			\tkzDefPoint(10,0){B}
			\tkzDefPoint(3.6,4.8){C}
			\tkzDrawPoints(A,B,C)
			\tkzLabelPoints(A,B)
			\tkzLabelPoint[above](C){$C$}
			\tkzDrawSegment(A,B)
			\tkzDrawSegment(B,C)
			\tkzDrawSegment(C,A)
			\tkzPicAngle["$\alpha$",draw=black,angle eccentricity=.6,angle radius=1cm](B,A,C)
			\tkzPicAngle["$\beta$",draw=black,angle eccentricity=.75,angle radius=1cm](C,B,A)
			\tkzPicAngle["$\gamma$",draw=black,angle eccentricity=.6,angle radius=1cm](A,C,B)
		\end{tikzpicture}
	\end{figure}
	
	\singletask[(\faBook\ S. 320)]{Geraden und Strecken}
	
	Zeichnen Sie eine Gerade $g$ und wählen Sie einen Punkt $A$ im Abstand von 3 cm zu $g$.
	Wählen Sie zusätzlich einen Punkt $B$ auf der Gerade so, dass die Strecke $\overline{AB}$ die Länge 4 cm hat.
	
	\checkered[9cm]
	
	\singletask[(\faBook\ S. 321)]{Vierecke}
	
	Zeichnen Sie...
	\begin{enumerate}[label=\alph*)]
		\item ...ein Rechteck, das kein Quadrat ist.
		\item ...ein Parallelogramm, das kein Rechteck ist.
	\end{enumerate}
	Berechnen Sie jeweils den Flächeninhalt und den Umfang des Vierecks.
	
	\checkered[5.5cm]
	
	\singletask[(\faBook\ S. 322)]{Dreiecke}
	
	Zeichnen Sie...
	\begin{enumerate}[label=\alph*)]
		\item ...ein stumpfwinkliges Dreieck.
		\item ...ein gleichschenkliges Dreieck.
		\item ...ein gleichseitiges Dreieck.
	\end{enumerate}
	Berechnen Sie jeweils den Flächeninhalt und den Umfang des Dreiecks.
	
	\checkered[5.5cm]
	
	\singletask[(\faBook\ S. 322--323)]{Satz des Pythagoras}
	
	Berechnen Sie die fehlende Seitenlänge $b$ mithilfe des Satz des Pythagoras.
	
	\begin{figure}[H]
		\centering
		\begin{tikzpicture}
			\tkzDefPoint(0,0){A}
			\tkzDefPoint(5,0){B}
			\tkzDefPoint(1.8,2.4){C}
			\tkzDrawPoints(A,B,C)
			\tkzLabelPoints(A,B)
			\tkzLabelPoint[above](C){$C$}
			\tkzDrawSegment(A,B)
			\tkzDrawSegment(B,C)
			\tkzDrawSegment(C,A)
			\tkzMarkRightAngle[german,size=.5,draw](A,C,B)
			\tkzLabelSegment(A,B){$c = 5\ \mathrm{cm}$}
			\tkzLabelSegment[above=.1cm, rotate=-35](B,C){$a = 4\ \mathrm{cm}$}
			\tkzLabelSegment[above left](A,C){$b$}
		\end{tikzpicture}
	\end{figure}
	
	\checkered[4.5cm]
	
	\singletask[(\faBook\ S. 323)]{Kreis}
	
	Zeichnen Sie einen Kreis mit Durchmesser 6 cm.
	Berechnen Sie den Radius, Umfang und Flächeninhalt des Kreises.
	
	\checkered[7cm]
\end{document}