\documentclass[11pt, a4paper, oneside]{article}
\usepackage[ngerman]{babel}
\usepackage{worksheet}

\begin{document}
	\author{L. Bung}
	\title{Python: Sets}
	\subject{Informatik}
	\class{FTE1}
	\maketitle
	
	\hint{Sets in Python}{
		Ähnlich wie Listen können wir in Python ein \textbf{Set} festlegen.
		Das Konzept des Sets ist aus mathematischer Perspektive eine Menge.
		
		Die Syntax ist exakt dieselbe wie bei den Listen, mit dem Unterschied, dass geschweifte Klammern statt eckigen Klammern verwendet werden:
		\lstinline[language=python]{set1 = \{1, 2, 3\}}
		
		Elemente können auch nachträglich zu einem Set hinzugefügt (\texttt{set1.add(element)}) oder aus einem Set entfernt werden (\texttt{set1.remove(element)}).
	}
	
	\partnertask{Eigenschaften von Sets}
	
	Legen Sie Sets und Listen an und vergleichen Sie diese miteinander.
	Füllen Sie dazu die folgende Tabelle aus.
	
	\begin{table}[H]
		\centering
		\begin{tabularx}{\textwidth}{| l | X | X |}
			\hline
			\textbf{Eigenschaft} & \textbf{Liste} & \textbf{Set}\\
			\hline
			\hline
			indizierbar? &&\\
			(\texttt{set1[0]} bzw. &&\\
			\texttt{liste1[0]})&&\\
			\hline
			Reihenfolge &&\\
			relevant? &&\\
			&&\\
			\hline
			Duplikate &&\\
			möglich? &&\\
			&&\\
			\hline
		\end{tabularx}
	\end{table}
	
	\pagebreak
	
	\partnertask{Mengenoperationen}
	
	Legen Sie zwei Sets \texttt{set1 = \{1, 2, 3, 4, \dq apple\dq, \dq banana\dq\}} und \texttt{set2 = \{3, 4, 5, 6, \dq banana\dq, \dq cherry\dq\}} an.
	
	Finden Sie heraus, wie die verschiedenen Operationen die beiden Sets miteinander verknüpfen:
	
	\begin{table}[H]
		\centering
		\begin{tabularx}{\textwidth}{| l | X |}
			\hline
			\textbf{Operation} & \textbf{Beschreibung}\\
			\hline
			\hline
			&\\
			\texttt{set1 - set2} &\\
			&\\
			\hline
			&\\
			\texttt{set1 \& set2} &\\
			&\\
			\hline
			&\\
			\texttt{set1 | set2} &\\
			&\\
			\hline
			&\\
			\texttt{set1 \char`\^\ set2} &\\
			&\\
			\hline
			&\\
			\texttt{set1 < set2} &\\
			&\\
			\hline
		\end{tabularx}
	\end{table}
	
	\singletask{Verschiedene Wörter in Sätzen}
	
	Schreiben Sie ein Programm, das einen Satz vom Nutzer einliest.
	Geben Sie dann alle eingegebenen Wörter, die \textit{unterschiedlich} sind, nacheinander aus.
	Geben Sie auch aus, wie viele Wörter es waren.
	
	Verwenden Sie zur Lösung ein Set.
\end{document}