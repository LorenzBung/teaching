\documentclass[11pt, a4paper, oneside]{article}
\usepackage[ngerman]{babel}
\usepackage{worksheet}

\begin{document}
	\author{L. Bung}
	\title{Python: Verzweigungen}
	\subject{Informatik}
	\class{FTE1}
	\maketitle

	Alle Programme, die wir bisher gesehen oder selbst geschrieben haben, waren relativ simpel gestrickt:
	Wir haben mithilfe von Variablen irgendwelche Formeln zur Berechnung neuer Werte angewendet.
	In der Regel sind Programme aber deutlich komplexer und müssen auf die Daten reagieren.
	
	\hint{Verzweigungen in Python}{
		Der Ablauf eines Programms kann in Python mit den Keywords \texttt{if}, \texttt{elif} und \texttt{else} beeinflusst werden.
		Abhängig von einer \emph{Bedingung} (englisch: condition) werden so bestimmte Teile des Codes ausgeführt oder auch nicht.
		\begin{itemize}
			\item Der Code innerhalb des \texttt{if}-Blocks wird nur ausgeführt, wenn die zugehörige Bedingung zutrifft.
			\item Der Code im \texttt{elif}-Block wird ausgeführt, wenn die Bedingung beim \texttt{if} nicht zutrifft, aber die Bedingung beim \texttt{elif} schon.
			\item Der Code im \texttt{else}-Block wird immer dann ausgeführt, wenn keine der vorherigen Bedingungen zugetroffen ist.
			Daher muss hier auch keine zusätzliche Bedingung angegeben werden.
		\end{itemize}
		Nach dem jeweiligen Keyword folgt die Bedingung (außer beim \texttt{else}) und anschließend ein Doppelpunkt.
		Innerhalb eines \texttt{if}-, \texttt{elif}- oder \texttt{else}-Blocks muss der Code eingerückt werden.
		Sowohl der \texttt{elif}-Block als auch der \texttt{else}-Block sind optional und können komplett weggelassen werden, wenn sie nicht benötigt werden.
	}
	
	Umgesetzt sähe das ganze beispielsweise folgendermaßen aus:
	\begin{lstlisting}[language=python]
alter = 0.5
if (alter > 18):
  # Wird ausgeführt, wenn alter > 18
  print("Nutzer ist volljährig!")
elif (alter < 1):
  # Wird ausgeführt, wenn alter <= 18 und alter < 1
  print("Nutzer ist ein Baby!")
else:
  # Wird ausgeführt, wenn alter <= 18 und alter >= 1
  print("Nutzer ist minderjährig!")
	\end{lstlisting}
	Dieser Code gibt beispielsweise ``Nutzer ist ein Baby!'' aus, denn die Bedingung in Zeile 2 trifft nicht zu, die Bedingung in Zeile 5 allerdings schon.
	
	\pagebreak
	\singletask{Teilbarkeit durch 5}
	
	Schreiben Sie ein Programm, das vom Nutzer eine Zahl einliest.
	Überprüfen Sie, ob es sich um eine durch 5 teilbare Zahl handelt und geben Sie abhängig davon einen entsprechenden Text aus.
	
	\hint{Logische Operatoren}{
		Wenn man das Ergebnis einer solchen Bedingung mal testweise in einer Variablen speichert, sieht man, dass es sich dabei um einen Boole'schen Wert handelt.
		\texttt{bedingung1 = 17 < 25} wäre beispielsweise \texttt{True}, während \texttt{bedingung2 = 25 < 17} den Wert \texttt{False} hätte.
		
		Man kann Bedingungen auch noch komplizierter gestalten, indem man sie mit Klammern und den Keywords \texttt{and}, \texttt{or} und \texttt{not} verbindet.
		Beispielsweise wäre \texttt{bedingung3 = (17 < 25) and (25 < 50)} zutreffend, da sowohl die erste als auch die zweite Klammer beide \texttt{True} sind.
	}
	
	So sähe beispielsweise eine Verzweigung mit etwas komplizierteren Bedingungen aus:
	
	\begin{lstlisting}[language=python]
alter = 13
groesse = 155
if (alter >= 14) and (groesse >= 150):
	print("Du darfst Achterbahn fahren")
elif not (alter >= 14) and (groesse >= 150):
	print("Du bist zwar groß genug, aber leider zu jung")
elif (alter >= 14) and not (groesse >= 150):
	print("Du bist zwar alt genug, aber leider zu klein")
else:
	print("Du bist leider zu jung und zu klein")
	\end{lstlisting}
	Die obige Person ist zwar groß genug, aber leider zu jung zum Achterbahn fahren.
	
	\singletask{Schaltjahrrechner}
	
	Schreiben Sie ein Programm, das prüft, ob ein vom Nutzer eingegebenes Jahr ein Schaltjahr ist:
	\begin{itemize}
		\item Ein Jahr ist Schaltjahr, wenn es durch 4 teilbar ist,
		\item aber nicht durch 100,
		\item außer es ist auch durch 400 teilbar.
	\end{itemize}
\end{document}