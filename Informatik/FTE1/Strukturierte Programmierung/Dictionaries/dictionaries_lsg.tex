\documentclass[11pt, a4paper, oneside]{article}
\usepackage[ngerman]{babel}
\usepackage{worksheet}

\begin{document}
	\author{L. Bung}
	\title{Python: Dictionaries \hspace{10cm} Lösungsvorschlag}
	\subject{Informatik}
	\class{FTE1-1}
	
	\date{21.01.2026}
	\maketitle
	
	\singletask{Personalausweis}
	
	Der folgende Lösungsvorschlag beinhaltet Aufgabe 1a), b) und c):
	
	\begin{lstlisting}[language=python]
# Teilaufgabe a)
personalausweis = {
	"vorname" : "Max",
	"nachname" : "Mustermann",
	"geschlecht" : 'm',
	# Ohne Teilaufgabe c)
	# "geburtsdatum" : "15.11.1989",
	# Mit Teilaufgabe c)
	"geburtsdatum" : {
		"tag" : 15,
		"monat" : 11,
		"jahr" : 1989
	},
	"koerpergroesse" : 181
}
# Teilaufgabe b)
personalausweis["nationalitaet"] = "deutsch"
personalausweis["augenfarbe"] = "blau"
for key in personalausweis.keys():
	print(key + ": " + str(personalausweis[key]))
	\end{lstlisting}
	
	\partnertask{Übersetzung des Maus-Intros}
	
	\lstinputlisting[language=python, firstline=129]{maus_lsg.py}
\end{document}