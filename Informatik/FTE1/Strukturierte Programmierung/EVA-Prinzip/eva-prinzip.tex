\documentclass[11pt, a4paper, oneside]{article}
\usepackage[ngerman]{babel}
\usepackage{worksheet}

\begin{document}
	\author{L. Bung}
	\title{EVA-Prinzip}
	\subject{Informatik}
	\class{FTE1}
	\maketitle
	
	\hint{Nutzereingaben in Python}{
		In Python kann der Nutzer des Programms mit dem Befehl \texttt{input()} um eine Eingabe gebeten werden.
		In den Klammern kann man zusätzlich noch als String einen Text übergeben, der dem Nutzer angezeigt wird.
		Anschließend kann die Eingabe in einer Variablen gespeichert werden.
		
		\emph{Achtung}: Die Eingabe des Nutzers wird immer als String gespeichert, auch wenn beispielsweise eine Kommazahl eingegeben wurde.
		Gegebenenfalls muss der Datentyp erst noch konvertiert werden.
	}
	
	In der Praxis sähe eine einfache Nutzereingabe so aus:
	\begin{lstlisting}[language=python]
name = input("Wie heißt du?")
print("Hallo " + name + "!")
	\end{lstlisting}
	
	\hint{EVA-Prinzip}{
		Eine typische Struktur eines Programms beinhaltet die folgenden Komponenten:
		\begin{enumerate}
			\item \emph{Eingabe}: Das Programm liest z.B. eine Datei ein oder Werte, die der Nutzer eingibt
			\item \emph{Verarbeitung}: Auf Basis der eingelesenen Daten passieren Dinge, beispielsweise werden die Daten modifiziert
			\item \emph{Ausgabe}: Zum Schluss wird etwas ausgegeben, z.B. eine bestimmte Meldung oder die veränderten Daten.
		\end{enumerate}
		Dieses Prinzip wird (nach den Anfangsbuchstaben der einzelnen Schritte) auch \textbf{EVA-Prinzip} genannt.
	}
	
	\singletask{Echo-Generator}
	
	Schreiben Sie ein Programm, das ein Wort und eine Zahl vom Nutzer abfragt.
	Geben Sie das Wort anschließend so häufig aus, wie der Nutzer eingegeben hat.
	Beispiel:
	\begin{table}[H]
		\centering
		\begin{tabular}{|l l l|}
			\hline
			\textbf{Eingabe Wort} & \textbf{Eingabe Zahl} & \textbf{Ausgabe}\\
			\hline
			Hallo & 5 & HalloHalloHalloHalloHallo\\
			Echo & 7 & EchoEchoEchoEchoEchoEchoEcho\\
			Hallo123 & 2 & Hallo123Hallo123\\
			\hline
		\end{tabular}
	\end{table}
	
	\singletask{BMI-Rechner}
	
	Schreiben Sie ein Programm, das die Größe und das Gewicht des Benutzers abfragt, den Body-Mass-Index (BMI) berechnet und den errechneten Wert ausgibt.
	
	\emph{Hinweis}: Die Formel zur Berechnung des BMI lautet $\mathrm{BMI} = \frac{\text{Gewicht in kg}}{\left(\text{Größe in m}\right)^2}$.
	
	\singletask{Promille-Rechner}
	
	Schreiben Sie ein Programm, das folgende Werte vom Nutzer einliest:
	\begin{itemize}
		\item das Gewicht des Nutzers (in kg)
		\item die Menge des getrunkenen Getränks (in Liter)
		\item der Alkoholgehalt des Getränks (in Prozent)
	\end{itemize}
	Berechnen Sie anschließend die Blutalkoholkonzentration des Nutzers\footnote{Das lässt sich z.B. mit der \href{https://de.wikipedia.org/wiki/Blutalkoholkonzentration\#Widmark-Formel}{Widmark-Formel} abschätzen}.
	Gehen Sie dazu wie folgt vor:
	
	\begin{enumerate}
		\item Berechnen Sie, wie viel Alkohol getrunken wurde: $$\text{Alkohol-Masse in g} = 10 \cdot \text{Getränkemenge in Liter} \cdot \text{Alkoholgehalt in Prozent} \cdot 0,8$$
		\item Berechnen Sie anschließend den Blutalkoholgehalt: $$\text{Blutalkoholkonzentration in Promille} = \frac{\text{Alkohol-Masse in g}}{\text{Masse der Person in kg} \cdot 0,7}$$
	\end{enumerate}
	
	Geben Sie den berechneten Wert anschließend aus.
\end{document}