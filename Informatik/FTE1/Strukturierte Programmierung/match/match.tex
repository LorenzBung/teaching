\documentclass[11pt, a4paper, oneside]{article}
\usepackage[ngerman]{babel}
\usepackage{worksheet}

\begin{document}
	\author{L. Bung}
	\title{Python: match}
	\subject{Informatik}
	\class{FTE1}
	\maketitle
	
	\hint{match-Blöcke}{
		Um eine Verzweigung mit vielen Bedingungen übersichtlicher zu schreiben, kann man in Python einen match-Block verwenden.
		
		Die zu überprüfende Bedingung wird hinter das Keyword \texttt{match} geschrieben.
		Die einzelnen Fälle können dann mit \texttt{case} angegeben werden.
		Mehrere Fälle können mit einem \texttt{|} verknüpft werden.
		
		``\texttt{case \_}'' ist der Fall, der ausgeführt wird, wenn keiner der vorhergehenden Fälle eingetreten ist (vergleichbar mit dem \texttt{else}-Block einer Verzweigung).
	}
			
	\begin{lstlisting}[language=python]
x = int(input("Bitte gib einen Wert für x an. "))
match x:
	case 1:
		print("Der Wert von x ist 1.")
	case 2 | 3:
  		print("Der Wert von x ist 2 oder 3.")
	case _:
		print("Der Wert von x ist weder 1, 2 oder 3.")
	\end{lstlisting}
	
	
	\singletask{Tiergeräusche}
	
	Schreiben Sie folgenden Code mit einem match-Block.
	
	\begin{lstlisting}[language=python]
geraeusch = input("Welches Geräusch macht das Tier? ")
if geraeusch == "miau":
	print("Katze")
elif geraeusch == "wuff":
	print("Hund")
elif geraeusch == "muh":
	print("Kuh")
else:
	print("Unbekannt")
	\end{lstlisting}
	
	\pagebreak
	
	\singletask{Schulnoten in Text umwandeln}
	
	Schreiben Sie ein Programm, das eine Schulnote (1–6) einliest und mit match einen passenden Text zurückgibt:
	
	\begin{itemize}
		\item 1: ``sehr gut''
		\item 2: ``gut''
		\item 3: ``befriedigend''
		\item 4: ``ausreichend''
		\item 5 oder 6: ``nicht bestanden''
		\item alles andere: ``ungültige Note''
	\end{itemize}
	
	\singletask{Zeichenklassifikation}
	
	Schreiben Sie ein Programm, das ein einzelnes Zeichen einliest und mit einem \texttt{match}-Block folgendermaßen klassifiziert:
	
	\begin{itemize}
		\item Geben Sie ``Vokal'' aus, wenn das Zeichen ein \texttt{a}, \texttt{e}, \texttt{i}, \texttt{o} oder \texttt{u} ist.
		\item Geben Sie ``Konsonant'' aus, wenn das Zeichen ein Buchstabe, aber kein Vokal ist.
		\item Geben Sie ``Zahl'' aus, wenn das Zeichen eine Ziffer zwischen \texttt{0} und \texttt{9} ist.
		\item Geben Sie ``Sonstiges'' aus, wenn keine der oben genannten Bedingungen zutrifft.
	\end{itemize}
	
	\singletask{Taschenrechner}
	
	Schreiben Sie ein Programm, das zwei Zahlen und ein Rechenzeichen (+, -, *, /) als String einliest.
	
	Nutzen Sie anschließend einen match-Block, um das entsprechende Ergebnis zu berechnen und auszugeben.
\end{document}