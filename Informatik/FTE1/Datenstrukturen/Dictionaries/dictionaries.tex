\documentclass[11pt, a4paper, oneside]{article}
\usepackage[ngerman]{babel}
\usepackage{worksheet}

\begin{document}
	\author{L. Bung}
	\title{Python: Dictionaries}
	\subject{Informatik}
	\class{FTE1-1}
	\date{21.01.2026}
	\maketitle
	
	\warning{Dictionaries}{
		Eine häufig verwendete Datenstruktur ist das sogenannte \textbf{Dictionary}.
		Dabei werden Paare von Werten -- bestehend aus einem \textbf{Schlüssel} und einem zugehörigen \textbf{Wert} -- einander zugewiesen.
		Die Schlüssel sind dabei eindeutig, Werte können aber doppelt vorkommen.
		\\
		
		Ein Dictionary wird in Python mit geschweiften Klammern angelegt, zwischen denen die Schlüssel-/Wert-Paare kommagetrennt stehen.
		Zwischen Schlüssel und Wert steht ein Doppelpunkt.
		Das könnte zum Beispiel so aussehen:
		
		\lstinline[language=python]{woerter = \{"eins" : 1, "zwei" : 2, "drei" : 3\}}
		\\
		
		Auf die einzelnen Elemente kann man nun ähnlich wie bei einer Liste zugreifen:
		
		\lstinline[language=python]{eins = woerter["eins"] #Der Wert der Variablen ist 1}
		\\
		
		Auch nach dem Anlegen kann man neue Werte hinzufügen und entfernen:
		
		\lstinline[language=python]{woerter["vier"] = 4 #Jetzt steht "vier" : 4 auch im Dictionary}
		
		\lstinline[language=python]{woerter.pop("eins") #Der Eintrag "eins" : 1 wird gelöscht}
		\\
		
		Die Schlüssel kann man auch als Liste verwenden (z. B. für die Verwendung in einer Schleife):
		
		\lstinline[language=python]{schluessel = woerter.keys() #schluessel = ["zwei", "drei", "vier"]}
	}
	
	\singletask[(\timeLimit{15 min.})]{Personalausweis}
	
	a) Legen Sie ein Dictionary an, das Sie in Form eines Steckbriefs beschreibt.
	Es soll die folgenden Schlüssel haben:
	
	\begin{itemize}
		\item Vorname
		\item Nachname
		\item Geschlecht
		\item Geburtsdatum
		\item Körpergröße
	\end{itemize}
	
	b) Erweitern Sie den angelegten Personalausweis nachträglich um Ihre Augenfarbe und Nationalität.
	Geben Sie zusätzlich alle Einträge des Ausweises aus.
	
	c) Legen Sie ein weiteres Dictionary an, das ein Datum speichern soll:
	Es soll die Schlüssel \texttt{day}, \texttt{month} und \texttt{year} haben.
	Haben Sie eine Idee, wie man dieses neue Dictionary im Personalausweis für das Geburtsdatum nutzen könnte?
	
	\pagebreak
	
	\partnertask[(\timeLimit{7 min.})]{Übersetzung des Maus-Intros}
	
	Laden Sie sich das vorgefertigte Python-Programm \texttt{maus.py} aus Moodle herunter.
	Darin sind bereits Variablen mit dem rumänischen Text sowie ein rumänisch-deutsches Wörterbuch angelegt.
	
	Schreiben Sie nun ein Programm, das den Text Wort für Wort durchgeht und übersetzt und die Übersetzung anschließend ausgibt
	\footnote{Die Python-Funktion \texttt{split()} zum Aufteilen eines Strings könnte hilfreich sein: \texttt{\dq Hallo 123\dq.split(\dq\ \dq)} wird zu einer Liste \texttt{[\dq Hallo\dq, \dq 123\dq]}.}.
	Können Sie mithilfe der Übersetzung sagen, worum es in der Folge wohl geht?
	
	\bonustask{Personalregister für die Klasse}
	
	Legen Sie mit Ihrer Lösung aus Aufgabe 1 Personalausweise für Ihre Mitschüler an.
	Speichern Sie diese in einem weiteren Dictionary, welches als Personalregister dient.
	
	Passen Sie das Programm so an, dass der Nutzer einen Namen angeben kann, zu dem er anschließend den entsprechenden Personalausweis ausgegeben bekommt.
\end{document}