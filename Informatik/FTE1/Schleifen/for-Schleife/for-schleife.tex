\documentclass[11pt, a4paper, oneside]{article}
\usepackage[ngerman]{babel}
\usepackage{worksheet}

\begin{document}
	\author{L. Bung}
	\title{Python: for-Schleifen}
	\subject{Informatik}
	\class{FTE1}
	\maketitle
	
	\hint{for-Schleifen}{
		In vielen Fällen müssen eine oder mehrere Befehle in einem Programm mehrfach ausgeführt werden -- beispielsweise, um jedes Element einer Liste auszugeben.
		Hierfür können Schleifen verwendet werden.
		
		Eine for-Schleife führt einen bestimmten Codeblock eine bestimmte Anzahl an Wiederholungen lang aus.
		Beispielsweise könnte man für eine Liste mit 4 Elementen eine for-Schleife mit 4 Wiederholungen verwenden, um jedes Element auszugeben.
		
		Die Ausführung der Schleife kann mit dem Keyword \texttt{break} abgebrochen werden.
	}
	
	Eine for-Schleife in Python sieht folgendermaßen aus:
	
	\begin{lstlisting}[language=python]
for i in range(1, 10):  #i hat Werte von 1 bis 9 (jeweils inklusive)
	print(i)  #Die Zahlen 0-9 werden ausgegeben.
	\end{lstlisting}
	
	Lässt man die erste Zahl bei \texttt{range()} weg, werden alle Zahlen ab (inklusive) der 0 genommen.
	
	\singletask{Zahlen ohne Ende}
	
	Schreiben Sie ein Programm, das eine Zahl vom Benutzer einliest und anschließend alle Zahlen von 0 bis zur eingegebenen Zahl ausgibt.
	
	\singletask{Gerade Zahlen bis 100}
	
	Schreiben Sie ein Programm, das alle geraden Zahlen bis 100 ausgibt.
	
	\singletask{Summe von vielen Zahlen}
	
	Schreiben Sie ein Programm, das 10 Zahlen vom Benutzer einliest und die Summe aller Zahlen berechnet.
	
	\hint{for-each-Schleife}{
		Eine for-each-Schleife wird nicht eine bestimme Anzahl an Wiederholungen lang ausgeführt, sondern einmal für jedes Element in einer Liste.
		Die Laufvariable ist dabei auch keine Zahl, sondern das Element der Liste selbst.
		
		Eigentlich sind for-Schleifen in Python gar keine echten for-Schleifen, wie man Sie aus anderen Programmiersprachen wie Java oder C++ kennt.
		In Wirklichkeit handelt es sich um for-each-Schleifen:
		\texttt{range(10)} ist nämlich eine Liste, die die Zahlen 0--9 enthält.
	}
	
	\singletask{Zahlen und Strings zählen}
	
	Sie haben folgende Liste als Variable gegeben:
	
	\begin{lstlisting}[language=python]
liste = ["A", 2, 4, "C", "I", 7, -2, "M", -14, "S", -9, 0, 412, "O", "J", -23, "L"]
	\end{lstlisting}
	
	Schreiben Sie ein Programm, das zählt, wie viele Strings und wie viele Zahlen in der Liste enthalten sind.
	
	\singletask{Nach Treffer abbrechen}
	
	Durchlaufen Sie eine Liste von Wörtern und geben Sie das erste Wort mit mehr als 5 Buchstaben aus.
	Brechen Sie danach die Schleife ab.
	
	Testen Sie Ihr Programm mit zwei verschiedenen Listen.
	
	\singletask{Duplikate entfernen}
	
	Gegeben ist eine Liste mit mehrfach vorkommenden Zahlen.
	Erstellen Sie mithilfe einer for-each-Schleife eine neue Liste, in der jedes Element nur einmal vorkommt und geben Sie die neue Liste anschließend aus.
	
	Testen Sie ihr Programm mit zwei verschiedenen Listen.
\end{document}