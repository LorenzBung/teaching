\documentclass[11pt, a4paper, oneside]{article}
\usepackage[ngerman]{babel}
\usepackage{worksheet}

\begin{document}
	\author{L. Bung}
	\title{Python: while-Schleifen}
	\subject{Informatik}
	\class{FTE1}
	\maketitle
	
	\hint{while-Schleifen}{
		Wir kennen bisher schon for-Schleifen, um Code wiederholt ausführen zu lassen.
		Diese Art von Schleifen sind dann besonders nützlich, wenn wir wissen, wie häufig etwas wiederholt werden soll.
		Wenn wir beispielsweise jedes Element einer Liste ausgeben wollen, müssen wir so häufig den Code ausführen, wie Elemente in der Liste sind.
		
		Manchmal ist noch nicht bekannt, wie häufig man den Code ausführen muss.
		Die while-Schleife führt den Code so lange aus, bis eine bestimmte Bedingung zutrifft.
		Diese Bedingung wird nach dem Keyword \texttt{while} angegeben.
	}
	
	In Python sähe das beispielsweise folgendermaßen aus:
	
	\begin{lstlisting}[language=python]
summe = 0
while summe < 10:
	eingabe = int(input("Bitte gib eine Zahl ein: "))
	summe += eingabe
print(summe)
	\end{lstlisting}
	
	Dieses Programm addiert so lange Zahlen (die der Nutzer eingibt) auf, bis die Summe über 10 liegt.
	Dann wird der Wert der Summe ausgegeben.
	
	\singletask{Wörter aneinanderhängen}
	
	Schreiben Sie ein Programm, das so lange Wörter vom Nutzer abfragt, bis dieser das Wort \texttt{exit} eingibt.
	Anschließend soll das Programm alle Wörter mit Leerzeichen aneinandergehängt ausgeben.
	
	\singletask{Wiederholte Multiplikation / Division}
	
	Schreiben Sie folgendes Programm:
	
	\begin{itemize}
		\item Setzen Sie zu Beginn eine Variable auf den Wert 10.
		\item Lesen Sie so lange Zahlen durch den Nutzer ein, bis die Variable unter 1 fällt oder über 100 steigt.
		\item Multiplizieren oder Dividieren Sie den Wert der Variablen abwechselnd mit der Nutzereingabe.
		\item Geben Sie zum Schluss den Wert der Variablen aus.
	\end{itemize}
	
	\singletask{Konvertierung von Dezimal nach Binär}
	
	Mit dem folgenden Verfahren kann man eine Dezimalzahl (z. B. $(43)_{10}$) in eine Binärzahl umwandeln:
	\begin{align*}
		43 : 2 &= 21 \text{ Rest } 1\\
		21 : 2 &= 10 \text{ Rest } 1\\
		10 : 2 &= 5 \text{ Rest } 0\\
		5 : 2 &= 2 \text{ Rest } 1\\
		2 : 2 &= 1 \text{ Rest } 0\\
		1 : 2 &= 0 \text{ Rest } 1\\
	\end{align*}
	Liest man die Reste von unten nach oben, ergibt sich die Binärzahl $(101011)_2$.
	
	Schreiben Sie ein Programm, das eine Dezimalzahl einliest und sie mit dem Verfahren in eine Dezimalzahl konvertiert und anschließend ausgibt.
	
	\textbf{Tipp 1}: Wenn Sie eine Liste verwenden, können Sie \lstinline[language=python]{liste.reverse()} verwenden.
	
	\textbf{Tipp 2}: Eine Liste kann zu einem String zusammengehängt werden: \lstinline[language=python]{" ".join(liste)} verknüpft die Einträge der Liste mit einem Leerzeichen.
	
	\singletask{Restwert eines Fahrrads\footnotemark}
	\footnotetext{\url{https://inf-schule.de/imperative-programmierung/python/konzepte/wiederholungen/wertverlust}}
	
	\begin{wrapfigure}{r}{.3\textwidth}
		\centering
		\includegraphics[width=.3\textwidth]{zuverkaufen.png}
	\end{wrapfigure}
	
	Handelt es sich hierbei um ein gutes Angebot, oder ist der Preis zu hoch?
	Mit Hilfe eines kleinen Programms lässt sich abschätzen, ob man das Fahrrad kaufen oder sich lieber nochmal nach einem besseren Angebot umsehen sollte.
	
	Schreiben Sie ein Programm, das den \texttt{neupreis} und das \texttt{alter} des Fahrrads vom Nutzer einliest.
	Anschließend soll der Restwert des Fahrrads bei einem Alter von 1, 2, 3, \dots Jahren errechnet und ausgegeben werden (bis zum tatsächlichen Alter des Fahrrads).
	
	Im ersten Jahr nach dem Kauf hat ein Rad einen Wertverlust von 30\%, ab dann verliert es jedes Jahr 20\% an Wert.
\end{document}