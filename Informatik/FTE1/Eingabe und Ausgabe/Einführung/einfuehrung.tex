\documentclass[11pt, a4paper, oneside]{article}
\usepackage[ngerman]{babel}
\usepackage{worksheet}

\begin{document}
	\author{L. Bung}
	\title{Python: Einführung}
	\class{FTE1}
	\subject{Informatik}
	\maketitle
	
	\partnertask{Was ist Python?}
	
	Finden Sie folgende Informationen heraus:
	
	a) Woher stammt der Name der Programmiersprache Python?
	
	\lines[3cm]
	
	b) In welchen Bereichen wird Python überall eingesetzt?
	
	\lines[3cm]
	
	c) Woran könnte diese weite Verbreitung liegen?
	
	\lines[3cm]
	
	
	\warning{Installation der benötigten Software}{
		Zum Programmieren von Python werden verschiedene Softwarekomponenten benötigt, welche auf den Schulrechnern bereits installiert sind.
		Sollten Sie auf Ihrem eigenen Gerät arbeiten, müssen Sie diese erst noch installieren.
		Falls Sie die Schulrechner verwenden, können Sie die beiden folgenden Aufgaben überspringen.
	}
	
	\singletask{Installation von Python 3}
	
	Installieren Sie auf Ihrem Rechner Python (Version $\geq 3$).
	Laden Sie dazu die aktuelle Version von der offiziellen Webseite\footnote{\url{https://www.python.org/downloads/}} herunter und installieren Sie diese.
	
	Um zu überprüfen, ob die Installation erfolgreich war, können Sie in der Eingabeaufforderung bzw. im Terminal den Befehl \texttt{python3 --version} aufrufen.
	
	
	\singletask{Installation einer IDE}
	
	Theoretisch könnte man nun beginnen, mit einem Texteditor Python-Programme zu schreiben.
	Eine IDE (Integrated Development Environment) ist ein Programm, was zusätzlich zum Editor weitere nützliche Funktionen bietet -- zum Beispiel kann das Programm direkt ausgeführt werden oder Fehler im Programmcode werden rot unterstrichen.
	
	Eine einfache IDE für die Python-Entwicklung ist Thonny.
	Laden Sie sich das Programm von der Webseite\footnote{\url{https://thonny.org/}} herunter und installieren Sie es.
	
	
	\singletask{Das erste Programm}
	
	Starten Sie Thonny und erstellen Sie ein neues Programm.
	Speichern Sie die Datei unter dem Namen \texttt{helloWorld.py}.
	
	In die Textdatei können Sie nun Ihren Programmcode schreiben.
	Starten Sie mit folgendem Programmcode:
	
	\begin{lstlisting}[language=python]
print("Hello world!")
	\end{lstlisting}
	
	a) Was passiert, wenn Sie den Code ausführen?
	
	\lines[1cm]
	
	b) Was passiert, wenn Sie den Text verändern? Können Sie die Klammern oder Anführungszeichen weglassen?
	
	\lines[1cm]
	
	c) Was verändert sich, wenn Sie an den Anfang der Zeile ein \texttt{\#} schreiben?
	
	\lines[2cm]
	
	d) Wie ist die Ausgabe, wenn Sie zwei Wörter in Anführungszeichen -- durch Komma getrennt -- zwischen die Klammern schreiben?
	
	\lines[2cm]
	
	e) Ändern Sie das Programm, so dass es Ihren Namen ausgibt und Sie kurz vorstellt.
	
	\hint{Kommentare in Python}{
		Kommentare sind Zeilen im Code, die nicht ausgeführt werden -- sie werden beim Programmablauf einfach ignoriert.
		Sie sind eine einfache und wichtige Möglichkeit, um bestimmte Codeteile zu erläutern.
		
		Einzeilige Kommentare können in Python mit einem \texttt{\#} am Anfang der Zeile gemacht werden.
		Ist ein ausführlicher, mehrere Zeilen überspannender Kommentar nötig, kann auch mit \texttt{\dq\dq\dq} der Kommentar begonnen und ebenfalls mit \texttt{\dq\dq\dq} wieder beendet werden.
	}
	
	\singletask{Python als Taschenrechner}
	
	Eine der Grundfunktionen jeder Programmiersprache sind arithmetische Operationen wie Addition, Multiplikation usw.
	
	a) Testen Sie die arithmetischen Operatoren \texttt{+, -, *, /} an zwei Zahlen.
	
	b) Können Sie herausfinden, was die Operatoren \texttt{//, **} und \texttt{\%} tun?
	
	\lines[3cm]
\end{document}