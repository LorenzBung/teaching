\documentclass[11pt, a4paper, oneside]{article}
\usepackage[ngerman]{babel}
\usepackage{worksheet}

\begin{document}
	\author{L. Bung}
	\title{Python: File Handling}
	\subject{Informatik}
	\class{FTE1}
	\maketitle
	
	\begin{hint}{Dateien öffnen, lesen/schreiben und schließen}
		Wollen wir in Python auf eine Datei zugreifen, erfolgt das in der folgenden Reihenfolge:
		\begin{enumerate}
			\item Die Datei wird geöffnet
			\item Es werden Daten gelesen oder geschrieben
			\item Die Datei wird geschlossen
		\end{enumerate}
		Zum Öffnen der Datei legen wir eine neue Variable an und rufen die Funktion \texttt{f = open(file, mode)} auf.
		Der Parameter \texttt{file} gibt dabei den Dateinamen an.
		Der Modus legt fest, ob wir nur lesen (\texttt{\dq r\dq}), die Datei überschreiben (\texttt{\dq w\dq}) oder an den Schluss der Datei etwas anhängen (\texttt{\dq a\dq}).
		
		Sobald die Datei geöffnet ist, kann man mit \texttt{f.read(n)} die nächsten \texttt{n} Zeichen oder mit \texttt{f.readline()} die nächste Zeile lesen.
		
		Haben wir die Datei im Modus \texttt{\dq w\dq} oder \texttt{\dq a\dq} geöffnet, können wir mit \texttt{f.write(string)} einen String in die Datei schreiben\footnotemark.
		
		Zum Schluss muss die Datei mit \texttt{f.close()} geschlossen werden.
		
		\begin{lstlisting}[language=python, numbers=none, frame=none]
f = open("log.txt", "a")  #Öffnet die Datei "log.txt"
print(f.readline())  #Gibt die erste Zeile der Logdatei aus
f.write("Logdatei wurde geöffnet\n")  #Schreibt eine Lognachricht ans Ende der Datei
f.close()  #Schließt die Datei "log.txt"
		\end{lstlisting}
	\end{hint}
	\footnotetext{Der String \texttt{\dq\textbackslash n\dq} gibt einen Zeilenumbruch an.}
	
	\singletask{Hello World und Echo}
	
	a) Schreiben Sie ein Programm, das eine neue Datei \texttt{hello.txt} mit dem Inhalt \texttt{Hello World!} anlegt.
	
	b) Schreiben Sie ein Programm, das einen Text vom Nutzer einliest und den Inhalt in die Datei \texttt{echo.txt} speichert.
	Bei erneuter Ausführung soll die Datei überschrieben werden.
	
	\pagebreak
	
	\singletask{To-Do-Liste}
	
	Schreiben Sie ein Programm, das eine To-Do-Liste erstellt (oder erweitert).
	Gehen Sie dafür folgendermaßen vor:
	
	\begin{itemize}
		\item Gibt der Nutzer \texttt{add [text]} ein, fügen Sie das neue To-Do \texttt{[text]} zur Datei \texttt{todo.txt} hinzu.
		\item Gibt der Nutzer \texttt{rm [text]} ein, markieren Sie das To-Do als erledigt (d. h.: löschen Sie die entsprechende Zeile aus der Datei \texttt{todo.txt}).
		\item Gibt der Nutzer \texttt{quit} ein, wird das Programm beendet.
	\end{itemize}
	
	\singletask{Notenverwaltung}
	
	Ein typisches Format zum Austausch von Daten sind \textbf{CSV-Dateien} (\textbf{c}omma/\textbf{c}haracter \textbf{s}eparated \textbf{v}alues).
	Die folgende CSV-Datei finden Sie auch in Moodle:
	
	\lstinputlisting[title=\texttt{noten.csv}]{noten.csv}
	
	\begin{enumerate}[label=\alph*)]
		\item Lesen Sie die CSV-Datei ein und geben Sie für jeden Schüler den Namen und die Noten schön formatiert aus.
		\item Berechnen Sie den Notendurchschnitt jedes Schülers und ergänzen Sie die Datei um eine Spalte \texttt{durchschnitt} mit dem entsprechenden Wert.
		\item Geben Sie den Vor- und Nachnamen des Schülers mit dem besten Schnitt aus.
	\end{enumerate}
\end{document}