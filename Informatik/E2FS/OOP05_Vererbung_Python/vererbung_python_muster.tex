\documentclass[11pt, a4paper, oneside]{article}
\usepackage{worksheet}

\begin{document}
	\author{L. Bung}
	\title{OOP: Vererbung (Python) \hspace{10cm} Musterlösung}
	\subject{SAE}
	\maketitle
	
	\singletask{Gartencenter-Produktverwaltung in Python}
	
	\lstinputlisting[language=python]{a1_muster.py}
	
	\pagebreak
	
	\partnertask{Richtung der Vererbung}

	a) Verändern Sie den Code, sodass die Richtung der Vererbung umgedreht wird.
	
	\begin{lstlisting}[language=python]
class GamingLaptop:
	def leistung(self):
		return "Sehr leistungsstark"
class Laptop(GamingLaptop):
	def leistung(self):
		return "Alltagstauglich"
	\end{lstlisting}
	
	b) Diskutieren Sie, welche Variante mehr Sinn ergibt.
	
	Die Variante, in der \texttt{GamingLaptop} von \texttt{Laptop} erbt, ergibt mehr Sinn:
	Ein Gaming-Laptop ist eine spezialisierte Variante eines ``normalen'' Laptops und nicht umgekehrt.
	Es könnte beispielsweise auch noch eine zweite Subklasse \texttt{ArbeitsLaptop} geben.
	
	Aus logischer Sicht ergibt das Codebeispiel aus der Aufgabe so wie es dasteht daher keinen Sinn.
	
\end{document}
