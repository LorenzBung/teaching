\documentclass[11pt, a4paper, oneside]{article}
\usepackage{worksheet}

\begin{document}
	\author{L. Bung}
	\title{OOP: Polymorphie}
	\subject{SAE}
	\maketitle
	
	\partnertask{Formen in Python}
	
	a) Setzen Sie folgendes UML-Diagramm in Python um.
	Die \texttt{area()}-Methode der Klasse \texttt{Shape} soll immer den Standardwert 0 zurückgeben.
	
	\begin{figure}[h]
		\centering
		\begin{tikzpicture}
			\begin{class}[text height=.5cm]{Shape}{4,3}
				\operation{+area(): float}
			\end{class}
			\begin{class}[text height=.5cm]{Rectangle}{0,0}
				\inherit{Shape}
				\attribute{-width: float}
				\attribute{-height: float}
				\operation{+area(): float}
			\end{class}
			\begin{class}[text height=.5cm]{Circle}{8,0}
				\inherit{Shape}
				\attribute{-radius: float}
				\operation{+area(): float}
			\end{class}
		\end{tikzpicture}
	\end{figure}
	
	b) Testen Sie Ihre Implementierung:
	Erstellen Sie jeweils ein Rechteck und einen Kreis und lassen Sie den Flächeninhalt berechnen.
	
	c) Ergänzen Sie die drei Klassen um eine Methode \texttt{circumference()}, die den Umfang der jeweiligen Form berechnet.
	
	d) Erweitern Sie Ihre Tests aus Teilaufgabe b): Erstellen Sie mehrere Objekte der beiden Klassen, speichern Sie sie in einer Liste und nutzen Sie anschließend eine Schleife, um die Methoden \texttt{area()} und \texttt{circumference()} aufzurufen.
	
\end{document}
