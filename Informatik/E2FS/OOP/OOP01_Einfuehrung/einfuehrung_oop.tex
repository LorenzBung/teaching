\documentclass[11pt, a4paper, oneside]{article}
\usepackage{worksheet}

\begin{document}
	\author{L. Bung}
	\title{Objektorientierte \hspace{10cm} Programmierung}
	\subject{SAE}
	\maketitle
	
	\singletask{Grundbegriffe der objektorientierten Programmierung}
	
	Ergänzen Sie folgende Tabelle um die fehlenden Einträge -- Objekte, Klassen, Attribute und Methoden.
	
	\begin{table}[h]
		\centering
		\setlength\tabcolsep{.25cm}
		\renewcommand{\arraystretch}{1.5}
		\begin{tabularx}{\textwidth}{|X|X|X|X|}
			\hline
			\textbf{Objekt} & \textbf{Klasse} & \textbf{Attribute} & \textbf{Methoden}\\
			\hline
			Raum 332\vspace{1cm} & & &\\
			\hline
			Mona Lisa\vspace{1cm} &&&\\
			\hline
			& Fußballverein\vspace{1cm} & &\\
			\hline
			&& Seitenanzahl, Titel, Autor\vspace{.5cm} &\\
			\hline
			&&&\texttt{bellen()}, \texttt{fressen()}, \texttt{stockholen()}\\
			\hline
		\end{tabularx}
	\end{table}

	\bonustask*{Bonusaufgabe}
	
	Überlegen Sie sich zusätzliche Beispiele für Objekte, die Sie aus Ihrem (Arbeits-)Alltag kennen.
	Finden Sie die zugehörigen Klassen, Attribute und Methoden.
	
	\pagebreak
	
	\hint{Klassen- und Objektdiagramme digital erstellen}{Es gibt viele verschiedene Anwendungen, welche zum Erstellen von Klassen- und Objektdiagrammen verwendet werden können.
	Eine einfache Möglichkeit hierzu bietet die Webseite \href{https://www.draw.io}{draw.io}.
	Dort gibt es bereits vorgefertigte Vorlagen für die beiden Diagrammtypen.}
	
	\singletask{Patientenverwaltung}
	
	Für ein Krankenhaus soll ein neues Softwaresystem entwickelt werden, was die Patientenverwaltung erleichtert.
	Erstellen Sie zur Planung der Anwendung ein Klassendiagramm.
	
	Zu Patienten werden Name, Geburtsdatum und Versicherungsnummer gespeichert.
	Ein Patient kann untersucht und entlassen werden.\\
	Weiterhin werden die Krankenzimmer mit Zimmernummer, Anzahl der Betten und zugehörigem Stationsarzt gespeichert.
	Die Zimmer können gereinigt und über das zentrale System auf- und abgeschlossen werden.\\
	Zusätzlich gibt es Karten, mit denen die Patienten kostenpflichtige Dienste wie Telefon, Internet oder die Cafeteria in Anspruch nehmen können.
	Eine Karte hat eine feste Kartennummer und ein Guthaben.
	Sie kann um einen bestimmten Betrag aufgeladen und entladen werden.
	
	\singletask{Test der Patientenverwaltung}
	
	Erstellen Sie nun ein Objektdiagramm.
	
	Beim ersten Test der Anwendung wird das Vierbettzimmer B310 unter der Aufsicht von Dr. Frank erstellt.
	Der Patient Paul Maier, geboren am 31.03.1985, wird außerdem in die Krankenhausakte aufgenommen.
	Er hat die Versicherungsnummer 5132781.
	Bei der Aufnahme wird ihm die Karte mit der Nummer 1 und einem Guthaben von 10€ ausgestellt.
	
	\pagebreak
	\partnertask{Erweiterung der Patientenverwaltung}
	
	Damit nachverfolgt werden kann, welcher Patient welche Karte ausgeliehen hat, soll auf der Karte der Patient gespeichert werden.
	
	a) Wie könnte man dies im Klassendiagramm modellieren?
	
	\lines[3cm]
	
	b) Passen Sie das Klassendiagramm aus Aufgabe 2 sowie das Objektdiagramm aus Aufgabe 3 entsprechend der neuen Änderung an.
	
	c) Zusätzlich soll bei Patienten ihr Krankenzimmer abgespeichert werden.
	Übernehmen Sie auch diese Änderung in die Diagramme.
	
	\bonustask{Diskussion}
	
	a) Überlegen Sie: Welche Vorteile hat die Modellierung mit Klassendiagrammen, anstatt direkt loszuprogrammieren?
	
	\lines[3cm]
	
	b) Welche Schwierigkeiten könnten sich bei der Modellierung komplexer Systeme ergeben?
	
	\lines[3cm]
	
\end{document}
