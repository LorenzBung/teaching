\documentclass[11pt, a4paper, oneside]{article}
\usepackage[ngerman]{babel}
\usepackage{worksheet}

\begin{document}
	\author{L. Bung}
	\title{Projekt: Datenbanken}
	\subject{SAE}
	\class{E2FI}
	\maketitle
	
	Dieses Softwareprojekt hat als Ziel, eine funktionsfähige Datenbank zu erstellen.
	Die Interaktion mit der Datenbank soll mithilfe eines Python-Programms stattfinden.
	Außerdem soll es möglich sein, die Datenbank als Datei zu exportieren und importieren.
	
	\section{Anforderungen an das Projekt}
	
	Inhaltlich gibt es keine Anforderungen an das Projekt.
	Sie können sich selbst entscheiden, welche Thematik behandelt wird -- beispielsweise ein Textadventure-Spiel, eine Datenbank für Sensor-Messwerte oder ein Ticketverwaltungssystem.
	
	Das fertige Projekt soll folgende Kriterien erfüllen:
	
	\begin{itemize}
		\item Es soll eine Datenbank mit mindestens 5 Tabellen vorliegen.
		\item Es sollen Datensätze in die Datenbank eingefügt werden können, ohne dass durch den Nutzer SQL-Code geschrieben werden muss.
		\item Es sollen Datensätze aus der Datenbank gelöscht werden können, ohne dass durch den Nutzer SQL-Code geschrieben werden muss.
		\item Der Nutzer soll zusätzlich die Möglichkeit haben, mithilfe von SQL-Abfragen selbst auf die Datenbank zuzugreifen (quasi als ``Debug-Modus'').
		Hierbei sollen auch Abfragen über mehrere Tabellen möglich sein.
		\item Es soll möglich sein, die aktuelle Datenbank als Datei im XML-, CSV- oder JSON-Format zu exportieren.
		\item Es soll möglich sein, einen bisherigen Export wieder in die Datenbank zu importieren.
		Hierbei soll die Datenbank durch die eingelesene Datei überschrieben werden.
	\end{itemize}
	
	
	\section{Bewertungskriterien}
	
	Das Projekt gliedert sich in drei Bestandteile:
	
	\begin{enumerate}
		\item Zur \textbf{Datenbank} gehörige Komponenten (SQL-Code, ER-Modell usw.)
		\item Das \textbf{Python-Programm}, welches der Interaktion mit der Datenbank dient
		\item Eine schriftliche \textbf{Dokumentation} des Projekts
	\end{enumerate}
	
	\subsection{Datenbank}
	
	Für die Datenbank spielen folgende Kriterien eine Rolle:
	
	\begin{itemize}
		\item Modellierung der Datenbank mithilfe eines ER-Modells
		\item Auflösung von Relationen unterschiedlicher Kardinalitäten
		\item Damit einhergehend: Überführung ins Relationenmodell
		\item Normalisierung der Datenbanktabellen in die 3. Normalform
		\item Implementierung der Datenbank mithilfe von SQL
	\end{itemize}
	
	\subsection{Python-Programm}
	
	Der Python-Code ist ausführlich zu dokumentieren.
	
	\subsection{Dokumentation}
	
	Sämtliche Arbeitsschritte sind in einer schriftlichen Projektdokumentation festzuhalten.
	Dazu zählen unter Anderem auch der Entwurf der Datenbank inklusive ER-Modell, aber auch beispielsweise der Aufbau des Python-Programms.
	
	Bitte geben Sie die fertige Dokumentation als PDF-Datei, nicht im \texttt{.docx}-Format o. Ä. ab.
	
	
	\section{Abgabe}
	
	Die Abgabe erfolgt als Upload via Moodle.
	Bitte beachten Sie die dort angegebene Deadline!
	Eine spätere Abgabe ist nur in Sonderfällen nach Rücksprache und vorheriger Genehmigung möglich.
	
	Sollten Sie in Ihrer Dokumentation bereits den SQL- und Python-Code anhängen, bitte ich dennoch um zusätzliche Abgabe der Codedateien.
\end{document}