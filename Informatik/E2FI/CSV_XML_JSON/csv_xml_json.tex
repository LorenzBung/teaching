\documentclass[11pt, a4paper, oneside]{article}
\usepackage[ngerman]{babel}
\usepackage{worksheet}

\begin{document}
	\author{L. Bung}
	\title{CSV / XML / JSON}
	\subject{SAE}
	\class{E2FI}
	\maketitle
	
	In vielen Fällen ist es hilfreich, Daten in eine Datei zu exportieren.
	Dies kann beispielsweise notwendig sein, um den Status des Programms zu speichern (z.B. den Spielstand in einem Computerspiel) oder um Daten zwischen verschiedenen Anwendungen auszutauschen.
	
	Drei häufige Dateiformate, die dabei verwendet werden, sind \textbf{CSV} (Comma-Separated Values), \textbf{XML} (Extensible Markup Language) und \textbf{JSON} (JavaScript Object Notation).
	
	\singletask{CSV (Comma-Separated Values)}
	
	Das CSV-Format ist ein relativ simples Format zur Speicherung von Informationen, die in tabellarischer Form vorliegen.
	Jede Tabellenzeile wird durch eine Zeile in der Datei repräsentiert, während die einzelnen Spalten innerhalb der Zeile durch Kommata getrennt werden.
	
	\warning{Trennzeichen bei CSV}{
		Es ist auch möglich, ein anderes Trennzeichen als das Komma bei CSV zu wählen -- beispielsweise ein Semikolon oder einen Backslash.
		Der Name CSV kann daher auch für Character-Separated Values stehen.
	}
	
	Ein typisches Beispiel für eine CSV-Datei sähe folgendermaßen aus:
	
	\begin{lstlisting}
bestell_id,kundenname,email,artikelname,preis,menge,bestelldatum
1001,Anna Meier,anna.meier@example.com,Laptop,899.99,1,2025-03-12
1001,Anna Meier,anna.meier@example.com,Maus,19.99,2,2025-03-12
1002,Paul König,paul.koenig@example.com,Tastatur,49.99,1,2025-03-13
1003,Anna Meier,anna.meier@example.com,Monitor,249.99,1,2025-03-14
	\end{lstlisting}
	
	a) Diskutieren Sie mögliche Vor- und Nachteile des CSV-Formats.
	
	\lines[5cm]
	
	b) Erstellen Sie ein ER-Modell der Situation.
	
	\boxarea[7cm]
	
	\bonustask*{Bonusaufgabe: CSV -- Umsetzung mit SQL}
	
	a) Geben Sie die SQL-Befehle zum Erstellen der Datenbanktabellen an.
	
	\lines[4cm]
	
	b) Geben Sie die SQL-Befehle an, um die in der CSV-Datei vorhandenen Datensätze in die Datenbank einzutragen.
	
	\lines[4cm]
	
	\pagebreak
	
	\singletask{XML (Extensible Markup Language)}
	
	Beim XML-Format werden die Daten zwischen einen öffnenden und schließenden Tag geschrieben.
	Die Tags werden mit spitzen Klammern ausgezeichnet; geschlossen wird ein Tag durch einen Schrägstrich innerhalb des Tags.
	Informationen können durch Schachtelung von Tags hierarchisch aufgebaut werden.
	
	Ein Beispiel für eine XML-Datei sieht so aus:
	
	\begin{lstlisting}[language=XML]
<buchladen>
  <buch>
  	<isbn>978-3-86680-192-9</isbn>
    <titel>Die Physiker</titel>
    <autoren>
      <autor>Friedrich Dürrenmatt</autor>
    </autoren>
    <preis>9.90</preis>
    <kategorie>Drama</kategorie>
  </buch>
  <buch>
  	<isbn>978-3-257-23011-9</isbn>
    <titel>Der Besuch der alten Dame</titel>
    <autoren>
      <autor>Friedrich Dürrenmatt</autor>
   	</autoren>
    <preis>10.90</preis>
    <kategorie>Drama</kategorie>
  </buch>
  <buch>
    <isbn>978-0-13-604259-4</isbn>
    <titel>Künstliche Intelligenz: Ein Moderner Ansatz</titel>
    <autoren>
      <autor>Peter Norvig</autor>
      <autor>Stuart Russell</autor>
    </autoren>
    <preis>45.50</preis>
    <kategorie>Lehrbuch</kategorie>
  </buch>
</buchladen>
	\end{lstlisting}
	
	\pagebreak
	
	a) Diskutieren Sie mögliche Vor- und Nachteile des CSV-Formats.
	
	\lines[5cm]
	
	b) Erstellen Sie ein ER-Modell der Situation.
	
	\boxarea[7cm]
	
	\bonustask*{Bonusaufgabe: XML -- Umsetzung mit SQL}
	
	a) Geben Sie die SQL-Befehle zum Erstellen der Datenbanktabellen an.
	
	\lines[5cm]
	
	b) Geben Sie die SQL-Befehle an, um die in der XML-Datei vorhandenen Datensätze in die Datenbank einzutragen.
	
	\lines[4cm]
	
	\singletask{JSON (JavaScript Object Notation)}
	
	Auch bei JSON handelt es sich um ein Dateiformat, das hierarchisch strukturiert werden kann.
	Die einzelnen Datensätze können als Schlüssel-Wert-Paar gespeichert werden -- analog zu den aus Python bekannten Dictionaries.
	Listen können durch kommagetrennte Angabe der Werte in eckigen Klammern ebenfalls ähnlich wie in Python genutzt werden.
	
	Eine typische JSON-Datei sieht folgendermaßen aus:
	
	\begin{lstlisting}[language=json]
{
  "lieferanten": [
    {"l_id": 1, "name": "PharmaPlus", "land": "Deutschland"},
    {"l_id": 2, "name": "Medicare GmbH", "land": "Österreich"}
  ], "medikamente": [
    {"m_id": 1, "name": "Ibuprofen 400mg", "preis": 4.50, "l_id": 1},
    {"m_id": 2, "name": "Paracetamol 500mg", "preis": 3.20, "l_id": 2}
  ], "bestellungen": [
    {"b_id": 1, "m_id": 1, "menge": 50, "datum": "2025-10-01"},
    {"b_id": 2, "m_id": 2, "menge": 30, "datum": "2025-10-03"}
  ]
}
	\end{lstlisting}
	
	a) Diskutieren Sie mögliche Vor- und Nachteile des JSON-Formats.
	
	\lines[4cm]
	
	b) Erstellen Sie ein ER-Modell der Situation.
	
	\boxarea[7cm]
	
	\bonustask*{Bonusaufgabe: JSON -- Umsetzung mit SQL}
	
	a) Geben Sie die SQL-Befehle zum Erstellen der Datenbanktabellen an.
	
	\lines[5cm]
	
	b) Geben Sie die SQL-Befehle an, um die in der JSON-Datei vorhandenen Datensätze in die Datenbank einzutragen.
	
	\lines[5cm]
	
	\bonustask{Konvertierung zwischen den Datentypen}
	
	a) Übersetzen Sie die CSV-Datei aus Aufgabe 1 zusätzlich ins XML- und JSON-Format.
	
	\boxarea[5.5cm]
	
	b) Übersetzen Sie die XML-Datei aus Aufgabe 2 zusätzlich ins CSV- und JSON-Format.
	
	\boxarea[5.5cm]
	
	c) Übersetzen Sie die JSON-Datei aus Aufgabe 3 zusätzlich ins CSV- und XML-Format.
	
	\boxarea[5.5cm]
	
\end{document}