\documentclass[11pt, a4paper, oneside]{article}
\usepackage[ngerman]{babel}
\usepackage{worksheet}

\begin{document}
	\author{L. Bung}
	\title{SQL: Aggregatsfunktionen}
	\subject{SAE}
	\class{E2FI}
	\maketitle
	
	\hint{Aggregatsfunktionen}{
		In SQL bezeichnet man bestimmte Funktionen, die aus einer Liste an Werten einen einzelnen Rückgabewert berechnen als \textbf{Aggregatsfunktionen}.
		
		Die häufigsten sind die folgenden:
		
		\begin{itemize}
			\item \texttt{MIN()}: Gibt den geringsten Wert zurück
			\item \texttt{MAX()}: Gibt den höchsten Wert zurück
			\item \texttt{COUNT()}: Zählt, wie viele Datensätze (bzw. Zeilen) vorliegen
			\item \texttt{SUM()}: Berechnet die Summe der Werte
			\item \texttt{AVG()}: Berechnet das arithmetische Mittel der Werte
		\end{itemize}
	}
	
	\singletask{Bestellungen}
	
	Sie haben folgende Tabelle gegeben:
	
	\begin{lstlisting}[language=sql]
bestellungen(id INT, kunde VARCHAR(50), produkt VARCHAR(50), anzahl INT, preis FLOAT, datum DATE)
	\end{lstlisting}
	
	\begin{enumerate}[label=\alph*)]
		\item Zählen Sie, wie viele Bestellungen insgesamt vorliegen.
		
		\lines[2cm]
		
		\item Zählen Sie, wie viele verschiedene Kunden bestellt haben.
		
		\lines[2cm]
		
		
		\item Zählen Sie, wie viele Bestellungen im Jahr 2025 getätigt wurden.
		
		\lines[2cm]
		
		\pagebreak
		
		\item Berechnen Sie den Gesamtumsatz.
		
		\lines[2cm]
		
		
		\item Ermitteln Sie für jeden Kunden den Gesamtumsatz.
		
		\lines[2cm]
		
		
		\item Finden Sie heraus, welches Produkt insgesamt am häufigsten verkauft wurde (nach Stückzahl).
		
		\lines[2cm]
		
		
		\item Berechnen Sie den durchschnittlichen Bestellwert.
		
		\lines[2cm]
		
		
		\item Bestimmen Sie den durchschnittlichen Preis pro Produkt.
		
		\lines[2cm]
		
		
		\item Ermitteln Sie den durchschnittlichen Umsatz pro Kunde.
		
		\lines[2cm]
		
		
		\item Finden Sie den niedrigsten und höchsten Einzelpreis eines Produkts.
		
		\lines[2cm]
		
		
		\item Zeigen Sie den ersten und letzten Bestelltermin im Datensatz.
		
		\lines[2cm]
	\end{enumerate}
\end{document}