\documentclass[11pt, a4paper, oneside]{article}
\usepackage[ngerman]{babel}
\usepackage{worksheet}

\begin{document}
	\author{L. Bung}
	\title{Einführung SQL}
	\subject{SAE}
	\class{E2FI}
	\maketitle
	
	\hint{Structured Query Language}{
		Bei \textbf{SQL} (Structured Query Language) handelt es sich um eine Datenbanksprache, die von mehr oder weniger allen gängigen Datenbanksystemen verstanden wird -- beispielsweise MySQL, PostgreSQL, MariaDB oder SQLite.
		Auch wenn die grundlegenden Keywords überall gleich sind, gibt es teilweise verschiedene ``Dialekte''.
	}
	
	Um eine Tabelle zu erstellen, verwendet man in SQL den Befehl \texttt{CREATE TABLE}.
	Dieser folgt dem folgenden Schema:
	
	\begin{lstlisting}[language=sql]
CREATE TABLE tabellenname(spalte1 DATENTYP, spalte2 DATENTYP, ...);
	\end{lstlisting}
	
	Beispielsweise erstellt der folgende Befehl eine Mitarbeiter-Tabelle:
	
	\begin{lstlisting}[language=sql]
CREATE TABLE Mitarbeiter(id INT, name VARCHAR(200), geburtsdatum DATE, gehalt FLOAT);
	\end{lstlisting}
	
	\hint{Datentypen in SQL}{
		Viele der Datentypen sind ähnlich zu den in Python verwendeten Typen.
		In vielen Fällen unterscheidet sich jedoch das dazugehörige Keyword:
		\begin{itemize}
			\item \texttt{CHAR(size)}: eine Zeichenkette fester Länge (z.B. Sozialversicherungsnummer)
			\item \texttt{VARCHAR(size)}: eine Zeichenkette variabler Länge (z.B. Name), jedoch maximal \texttt{size} Zeichen
			\item \texttt{BOOL}: Boole'scher Wert
			\item \texttt{INT}: Ganzzahlige Werte
			\item \texttt{FLOAT} bzw. \texttt{DOUBLE}: Gleitkommazahlen
			\item \texttt{DATE}: Ein Datum im Format \texttt{YYYY-MM-DD}
		\end{itemize}
		Eine vollständige Liste der Datentypen findet sich online\footnotemark.
	}
	\footnotetext{\url{https://www.w3schools.com/sql/sql_datatypes.asp}}

	Versehentlich erstellte oder nicht mehr benötigte Tabellen können mit dem Befehl \texttt{DROP TABLE} gelöscht werden:
	
	\begin{lstlisting}[language=sql]
DROP TABLE mitarbeiter;
	\end{lstlisting}
	
	\pagebreak
	\singletask{Tabellen erstellen}
	
	Auf dem Arbeitsblatt ``Einführung Datenbanken'' haben Sie sich bereits Gedanken gemacht, wie Sie Produkte und Kunden als Tabelle speichern könnten.
	
	a) Geben Sie jeweils den SQL-Befehl zur Erstellung der zugehörigen Tabellen an.
	
	\boxarea[5cm]
	
	\hint{Null-Werte verhindern}{
		In der Informatik werden leere (z. B. nicht gesetzte) Werte auch als Null-Werte bezeichnet.
		In SQL kann dies vermieden werden, indem man nach dem Datentyp das Keyword \texttt{NOT NULL} angibt.
		So wird erzwungen, dass der Wert gesetzt werden muss.
	}
	
	\hint{Primärschlüssel}{
		Um sicherzustellen, dass jeder Datensatz in der Tabelle eindeutig identifizierbar ist (nicht wie beim Beispiel \emph{Thomas Müller}), sollte bei jeder Tabelle ein Attribut eindeutig pro Datensatz sein.
		Dieses Attribut nennt man \textbf{Primärschlüssel}.
		In SQL lässt sich das Attribut durch Einfügen des Keywords \texttt{PRIMARY KEY} nach dem Datentyp festlegen.
	}
	
	b) Geben Sie für die Tabellen einen passenden Primärschlüssel an.
	Verändern Sie die SQL-Befehle zur Erstellung der Tabellen um sinnvolle Primärschlüssel und \texttt{NOT NULL}-Einschränkungen.
	
	\boxarea
	
	\singletask{Datensätze einfügen}
	
	Nachdem wir eine Tabelle erstellt haben, können wir nun einzelne Datensätze einfügen.
	Dies geschieht mit dem Keyword \texttt{INSERT} nach folgendem Schema:
	\begin{lstlisting}[language=sql]
INSERT INTO tabellenname(spalte1, spalte2, ...) VALUES (wert1, wert2, ...);
	\end{lstlisting}
	
	Beispielsweise legen die folgenden Befehle eine Tabelle ``Mitarbeiter'' und einen Datensatz in dieser an:
	\begin{lstlisting}[language=sql]
INSERT INTO mitarbeiter(id, name, geburtsdatum, gehalt) VALUES (0001,   'Thomas Müller', '1980-03-25', 4250.00);
	\end{lstlisting}
	
	a) Legen Sie für die beiden Tabellen aus Aufgabe 1 mehrere Datensätze an.
	Geben Sie jeweils den zugehörigen SQL-Befehl an.
	
	\boxarea[7cm]
	
	b) Was passiert, wenn Sie einen Datensatz mit fehlerhaften Werten anlegen wollen?
	Wie sieht es aus, wenn Sie den Primärschlüssel doppelt vergeben?
	
	\boxarea
	
\end{document}