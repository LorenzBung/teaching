\documentclass[11pt, a4paper, oneside]{article}
\usepackage[ngerman]{babel}
\usepackage{worksheet}

\begin{document}
	\author{L. Bung}
	\title{Relationenmodell}
	\subject{SAE}
	\class{E2FI2}
	\date{16.10.2025}
	\maketitle
	
	\partnertask[(\timeLimit{15 min.})]{Vom ER-Modell zur Tabelle}
	
	Das folgende ER-Modell soll in Tabellenform umgesetzt werden.
	Suchen Sie gemeinsam nach einer Möglichkeit, um die verschiedenen Relationen tabellarisch darstellen zu können.
	
	\begin{figure}[H]
		\centering
		\begin{tikzpicture}[node distance=3cm]
			\node[entity] (autor) {Autor};
			\node[attribute] (autorid) [left of=autor] {\underline{AutorID}} edge (autor);
			\node[attribute] (autorname) [above right of=autor] {Name} edge (autor);
			\node[attribute] (geburtsdatum) [above left of=autor] {Geburtsdatum} edge (autor);
			\node[relationship] (schreibt) [right of=autor] {schreibt} edge node[above]{$n$} (autor);
			\node[entity] (publikation) [right of=schreibt] {Publikation} edge node[above]{$m$} (schreibt);
			\node[attribute] (publikationsid) [above right of=publikation] {\underline{PublikationsID}} edge (publikation);
			\node[attribute] (publikationstitel) [right of=publikation] {Titel} edge (publikation);
			\node[relationship] (veröffentlicht) [below of=publikation, align=center] {veröffentlicht\\ in} edge node[left]{$n$} (publikation);
			\node[entity] (journal) [below of=veröffentlicht] {Journal} edge node[left]{$1$} (veröffentlicht);
			\node[attribute] (journalid) [right of=journal] {\underline{JournalID}} edge (journal);
			\node[attribute] (journaltitel) [below right of=journal] {Titel} edge (journal);
			\node[relationship] (hat) [below of=autor] {hat} edge node[right]{$1$} (autor);
			\node[entity] (login) [below of=hat] {Login} edge node[right]{$1$} (hat);
			\node[attribute] (kontonummer) [below left of=login] {\underline{Kontonummer}} edge (login);
			\node[attribute] (passwort) [below right of=login] {Passwort} edge (login);
		\end{tikzpicture}
	\end{figure}
	
	\lines[6cm]
	
	\singletask*[(\timeLimit{10 min.})]{Aufgabe 2}
	
	Schreiben Sie das folgende ER-Modell in Relationenschreibweise.
	Überlegen Sie sich dazu, wie Sie die Relationen geeignet auflösen können.
	
	\begin{figure}[H]
		\centering
		\begin{tikzpicture}[node distance=2.5cm]
			\node[entity] (lehrkraft) {Lehrkraft};
			\node[attribute] (kürzel) [right of=lehrkraft] {\underline{Kürzel}} edge (lehrkraft);
			\node[attribute] (lehrkraftname) [left of=lehrkraft] {Name} edge (lehrkraft);
			\node[relationship] (unterrichtet) [below of=lehrkraft] {unterrichtet} edge node[right]{$1$} (lehrkraft);
			\node[entity] (kurs) [below of=unterrichtet] {Kurs} edge node[right]{$n$} (unterrichtet);
			\node[relationship] (belegt) [below left of=kurs] {belegt} edge node[below right]{$m$} (kurs);
			\node[attribute] (kursID) [left of=kurs] {\underline{KursID}} edge (kurs);
			\node[attribute] (kursBezeichnung) [right of=kurs, node distance=3.5cm] {Bezeichnung} edge (kurs);
			\node[entity] (schüler) [below left of=belegt] {Schüler} edge node[above left]{$n$} (belegt);
			\node[attribute] (schülerID) [above left of=schüler] {\underline{SchülerID}} edge (schüler);
			\node[attribute] (schülername) [left of=schüler] {Name} edge (schüler);
			\node[relationship] (findetstatt) [below right of=kurs, align=center] {findet\\ statt in} edge node[below left]{$n$} (kurs);
			\node[entity] (raum) [below right of=findetstatt] {Raum} edge node[above right]{$1$} (findetstatt);
			\node[attribute] (raumnummer) [right of=raum, node distance=3cm] {\underline{Nummer}} edge (raum);
			\node[attribute] (raumkapazität) [above right of=raum] {Kapazität} edge (raum);
		\end{tikzpicture}
	\end{figure}
	
	\lines[9cm]
	
	\hint{Relationenschreibweise}{
		Die Relationenschreibweise wird dazu verwendet, um einfach darzustellen, wie die (mithilfe eines ER-Modells) modellierte Datenbank nun tatsächlich umzusetzen ist.
		Es werden der Name der Tabelle sowie aller Tabellenspalten angegeben.
		Primär- und Fremdschlüssel werden dabei gekennzeichnet -- Primärschlüssel unterstrichen, Fremdschlüssel gestrichelt unterstrichen bzw. gelegentlich auch kursiv gedruckt.
		
		Das Relationenmodell für einen Kunden mit dem Primärschlüssel \texttt{KundenID} sähe beispielsweise folgendermaßen aus:
		
		\texttt{Kunde(\underline{KundenID}, Name, Geburtsdatum, \dashuline{Ausweisnummer})}
		
		Der Fremdschlüssel \texttt{Ausweisnummer} verweist auf den Primärschlüssel einer zweiten Tabelle (für den Ausweis).
	}
	
	\hint{Überführung von Relationen verschiedener Kardinalitäten}{
	Je nachdem, um was für eine Relation es sich handelt, müssen diese unterschiedlich aufgelöst werden.
	
	\begin{itemize}
		\item \textbf{1:1-Relation}: Fremdschlüssel in einer der beiden Tabellen mit Verweis auf die andere Tabelle.
		Es kann gewählt werden, in welcher der beiden Tabellen der Fremdschlüssel gespeichert wird.
		\item \textbf{1:n-Relation}: Fremdschlüssel auf der n-Seite mit Verweis auf die 1-Seite.
		\item \textbf{n:m-Relation}: Auflösen durch Extra-Tabelle nötig.
		Dort werden die beiden Fremdschlüssel der n- und m-Seite gespeichert.
	\end{itemize}
	}
	
	\singletask*[(\timeLimit{5 min.})]{Aufgabe 3: Quiz}
	
	\begin{minipage}[t]{.75\textwidth}
		\vspace{0pt}
		Überprüfen Sie Ihr Wissen mithilfe der LearningApp.
	\end{minipage}
	\begin{minipage}[t]{.25\textwidth}
		\vspace{0pt}
		\qrlink{https://learningapps.org/watch?v=ps7diyz3t25}
	\end{minipage}
	
	
	\bonustask*{Bonusaufgabe}
	
	Erweitern Sie das ER-Modell aus Aufgabe 2:
	Es soll die Möglichkeit geben, dass Schülerinnen und Schüler Noten erhalten.
	
	Setzen Sie anschließend die daraus entstandene Relation mit der Relationenschreibweise um.
\end{document}