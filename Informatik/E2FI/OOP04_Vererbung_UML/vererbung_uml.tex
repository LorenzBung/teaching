\documentclass[11pt, a4paper, oneside]{article}
\usepackage{worksheet}

\begin{document}
	
	\makeheader{SAE}{OOP: Vererbung (UML)}
	
	\partnertask{Produktverwaltung eines Gartencenters}
	
	Ein Gartencenter verwaltet ihre Produkte mithilfe eines Softwaresystems, welches überarbeitet werden soll.
	Das System arbeitet aktuell mit den Klassen \textbf{Pflanze}, \textbf{Blumentopf} und \textbf{Werkzeug}.
	
	Sowohl Pflanzen, Blumentöpfe als auch Werkzeuge verfügen über eine Produktnummer, eine Bezeichnung und einen Preis.
	Außerdem kann jedes Produkt dieser Kategorien verkauft werden.
	
	a) Ergänzen Sie die UML-Klassendiagramme um die oben beschriebenen Informationen.
	Fügen Sie zusätzlich sinnvolle Attribute und Methoden zu den einzelnen Klassen hinzu.
	
	\begin{figure}[h]
		\centering
		\vspace{5cm}
		\begin{tikzpicture}
			\begin{class}[text width=4cm]{Pflanze}{0,0}
				\attribute{}
				\attribute{}
				\attribute{}
				\attribute{}
				\attribute{}
				\operation{}
				\operation{}
				\operation{}
				\operation{}
			\end{class}
			\begin{class}[text width=4cm]{Blumentopf}{5,0}
				\attribute{}
				\attribute{}
				\attribute{}
				\attribute{}
				\attribute{}
				\operation{}
				\operation{}
				\operation{}
				\operation{}
			\end{class}
			\begin{class}[text width=4cm]{Werkzeug}{10,0}
				\attribute{}
				\attribute{}
				\attribute{}
				\attribute{}
				\attribute{}
				\operation{}
				\operation{}
				\operation{}
				\operation{}
			\end{class}
		\end{tikzpicture}
	\end{figure}
	
	b) Welche Probleme stellen Sie fest? Haben Sie Ideen, wie man diese lösen könnte?
	
	\lines[3cm]
	
	\singletask{Erweiterung des Gartencenter-Systems}
	
	Zusätzlich zu den verkauften Produkten bietet die Gärtnerei auch noch \textbf{Dienstleistungen} (zum Beispiel Baumbeschneidungen) und \textbf{Veranstaltungen} (zum Beispiel Kurse zur Gartenpflege) an.
	
	Entwerfen Sie ein UML-Klassendiagramm\footnote{z.B. wieder mithilfe von \url{https://draw.io}} mit mindestens einer Oberklasse und mindestens zwei Unterklassen, bei denen Vererbung sinnvoll eingesetzt wird.
	Überlegen Sie sich sinnvolle Attribute und Methoden für jede Klasse.
	
	\boxarea[7cm]
	
	\partnertask{Diskussion und Reflexion}
	
	a) Welche Vorteile erkennen Sie beim Einsatz von Vererbung?
	
	\lines[3cm]
	
	b) Können Sie sich Fälle vorstellen, in es zu Problemen kommen könnte?
	
	\lines[3cm]
	
	\singletask*{Arbeitsauftrag für die 2. Stunde}
	
	1. Finden Sie ein Beispiel aus Ihrem Arbeitsalltag, in dem Vererbung eine Rolle spielt.
	Erstellen Sie ein einfaches UML-Klassendiagramm, was die Situation darstellt.
	
	\boxarea[10cm]
	
	2. Einige Aufgaben aus den letzten Stunden konnten aus Zeitmangel nicht zu Ende bearbeitet (oder gar nicht erst angefangen) werden.
	Nutzen Sie die restliche Zeit, um Aufgaben zu den Themen nachzuholen, bei denen Sie die größten Schwierigkeiten hatten.
	
	\textbf{Bitte laden Sie Ihre Ergebnisse am Ende der Stunde auf Moodle in den Studierendenordner ``Ergebnisse Arbeitsauftrag'' hoch.}
	
\end{document}
