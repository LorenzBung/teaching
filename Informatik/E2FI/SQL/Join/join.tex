\documentclass[11pt, a4paper, oneside]{article}
\usepackage[ngerman]{babel}
\usepackage{worksheet}

\begin{document}
	\author{L. Bung}
	\title{SQL: Joins}
	\subject{SAE}
	\class{E2FI}
	\maketitle
	
	In vielen Fällen ist es nötig, Informationen aus mehreren Tabellen gleichzeitig anzuzeigen oder z. B. für komplexere Bedingungen miteinander zu verbinden.
	Insbesondere bei gut entworfenen Datenbanken kommt man eigentlich nicht darum herum.
	
	\hint{Abfragen über mehrere Tabellen}{
		Im klassischen \texttt{SELECT}/\texttt{FROM}/\texttt{WHERE} können auch mehrere Tabellen abgefragt werden.
		Dafür geht man wie folgt vor:
		\begin{itemize}
			\item Beim \texttt{SELECT} gibt man die Spaltennamen wie gewohnt an.
			Haben die beiden Tabellen den selben Spaltennamen, schreibt man davor (durch Punkt getrennt) den Tabellennamen.
			\item Hinter dem \texttt{FROM} müssen die beiden Tabellennamen durch Komma getrennt angegeben werden.
			\item Beim \texttt{WHERE} können Bedingungen wie gewohnt gemacht werden.
		\end{itemize}
		Damit man nicht so viel Schreibarbeit hat, können mit dem Schlüsselwort \texttt{AS} Kurzbezeichnungen für Tabellen vergeben werden.
	}
	
	Für die beiden Tabellen \texttt{Student(\underline{id}, \dashuline{kursid}, vorname, nachname, gebdatum)} und \texttt{Kurs(\underline{id}, bezeichnung, dozent, raum)} könnte man beispielsweise den folgenden Befehl formulieren:
	
	\begin{lstlisting}[language=sql]
SELECT s.id, k.id
FROM Student AS s, Kurs AS k
WHERE s.kursid = k.id
	\end{lstlisting}
	
	Der Befehl gibt die IDs von Studenten und Kursen aus, die jeweils zusammen gehören.
	
	\warning{Joins}{
		Gibt man mehrere Tabellen durch Komma getrennt an, wird das sogenannte Kreuzprodukt verwendet:
		Alle Einträge der einen Tabelle werden mit allen Einträgen der anderen Tabelle verknüpft.
		
		Das ist in der Regel sehr ineffizient.
		Hier kommt ein sogenannter \textbf{JOIN} ins Spiel:
		Dabei werden nur noch Einträge verknüpft, bei denen eine bestimmte Bedingung zutrifft.
		
		Im SQL-Befehl wird statt \texttt{FROM Tabelle1, Tabelle2} angegeben:
		
		\texttt{FROM Tabelle1 JOIN Tabelle2 ON Tabelle1.id = Tabelle2.id} (im Fall, dass die Tabellen-IDs gleich sein sollen).
	}
	
	\pagebreak
	
	\singletask{Left, Right, Full Join}
	
	Recherchieren Sie die Unterschiede zwischen den verschiedenen Joins:
	
	\begin{itemize}
		\item \texttt{INNER JOIN}
		\item \texttt{LEFT JOIN}
		\item \texttt{RIGHT JOIN}
		\item \texttt{FULL JOIN}
	\end{itemize}
	
	\lines[6cm]
	
	\singletask{Mehrtabellen-Abfragen}
	
	Gegeben sind die folgenden Tabellen:
	
	\begin{itemize}
		\item \texttt{Student(\underline{stud\_id}, name, semester, \dashuline{fachbereich\_id})}
		\item \texttt{Fachbereich(\underline{fachbereich\_id}, name)}
		\item \texttt{Kurs(\underline{kurs\_id}, titel, \dashuline{lehrkraft\_id}, ects)}
		\item \texttt{Lehrkraft(\underline{lehrkraft\_id}, name, rang)}
		\item \texttt{Belegung(\underline{\dashuline{stud\_id}, \dashuline{kurs\_id}}, note)}
	\end{itemize}
	
	\begin{enumerate}[label=\alph*)]
		\item Listen Sie alle Kurse zusammen mit dem Namen der jeweiligen Lehrkraft auf.
		
		\lines[4cm]
		
		\item Geben Sie die Namen aller Studierenden aus, die Kurse bei einer Lehrkraft mit dem Rang ``Professor'' belegt haben.
		
		\lines[3cm]
		
		\item Zeigen Sie alle Kurse an, die mehr als 5 ECTS haben, inklusive Lehrkraftnamen.
		
		\lines[3cm]
		
		\item Geben Sie alle Studierenden und zugehörigen Kursnoten aus, sortiert nach Note (die besten zuerst).
		
		\lines[3cm]
		
		\item Ermitteln Sie zu jedem Kurs die Anzahl der Teilnehmenden.
		
		\lines[3cm]
		
		\item Zeigen Sie je Fachbereich die durchschnittliche Note aller zugehörigen Studierenden.
		
		\lines[2cm]
		
		\item Listen Sie für jede Lehrkraft die Gesamtanzahl der ECTS, die sie unterrichtet.
		
		\lines[3cm]
		
		\item Finden Sie alle Studierenden, die mehr als drei Kurse belegt und einen Notenschnitt besser als 2,0 haben.
		
		\lines[3cm]
		
		\item Ermitteln Sie alle Lehrkräfte, deren Kurse im Durchschnitt eine bessere Note als 2,5 erhalten.
		
		\lines[3cm]
		
		\item Geben Sie alle Fachbereiche aus, deren Studierende mindestens zwei unterschiedliche Lehrkräfte besucht haben.
		
		\lines[3cm]
		
		\item Geben Sie für jede Lehrkraft die Anzahl der Studierenden aus, die bei ihr mindestens zwei Kurse belegt haben.
		
		\lines[3cm]
		
		\item Finden Sie alle Studierenden, die in jedem belegten Kurs besser als 2,0 waren.
		
		\lines[3cm]
	\end{enumerate}
\end{document}