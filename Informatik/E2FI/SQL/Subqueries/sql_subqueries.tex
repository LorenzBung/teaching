\documentclass[11pt, a4paper, oneside]{article}
\usepackage[ngerman]{babel}
\usepackage{worksheet}

\begin{document}
	\author{L. Bung}
	\title{SQL: Subqueries}
	\subject{SAE}
	\class{E2FI}
	\maketitle
	
	\hint{Subqueries}{
		Manchmal ist es nötig, mehrere SQL-Abfragen ineinander zu schachteln.
		Diese Art von Abfragen nennt man \textbf{Subqueries} -- also Unterabfragen.
		
		Subqueries können an mehreren Stellen stehen:
		\begin{itemize}
			\item im \texttt{SELECT}: z. B. um eine spezielle Spalte auszugeben
			\item im \texttt{FROM}: für Abfragen auf einer Untertabelle
			\item im \texttt{WHERE}: Um Ergebnisse zu filtern
			\item im \texttt{HAVING}: Um Gruppen zu filtern
		\end{itemize}
		
		Subqueries werden in Klammern geschrieben und haben kein eigenes Semikolon.
	}
	
	Beispiel: Der folgende SQL-Befehl gibt die Vornamen aller Mitarbeiter aus, die mehr als der Durchschnitt verdienen.
	
	\begin{lstlisting}[language=sql]
SELECT first_name
FROM employees
WHERE salary > (SELECT AVG(salary) FROM employees);
	\end{lstlisting}
	
	\singletask{Verschiedene Subqueries}
	
	Gegeben sind die folgenden Tabellen:
	
	\begin{itemize}
		\item \texttt{Student(\underline{stud\_id}, name, semester, \dashuline{fachbereich\_id})}
		\item \texttt{Fachbereich(\underline{fachbereich\_id}, name)}
		\item \texttt{Kurs(\underline{kurs\_id}, titel, \dashuline{lehrkraft\_id}, ects)}
		\item \texttt{Lehrkraft(\underline{lehrkraft\_id}, name, rang)}
		\item \texttt{Belegung(\underline{\dashuline{stud\_id}, \dashuline{kurs\_id}}, note)}
	\end{itemize}
	
	\begin{enumerate}[label=\alph*)]
		\item Geben Sie alle Studierenden aus, deren Semester über dem Durchschnittssemester aller Studierenden liegt.
		
		\lines[4cm]
		
		\item Geben Sie alle Kurse aus, deren ECTS gleich dem höchsten ECTS-Wert aller Kurse ist.
		
		\lines[4cm]
		
		\item Geben Sie alle Lehrkräfte aus, deren durchschnittliche ECTS-Zahl pro Kurs über dem globalen Durchschnitt aller Kurse liegt.
		
		\lines[4cm]
		
		\item Geben Sie alle Fachbereiche aus, deren beste Note (Minimum) gleich der besten Note der gesamten Datenbank ist.
		
		\lines[4cm]
		
		\item Geben Sie alle Studierenden aus, deren persönlicher Notenschnitt besser ist als der Notenschnitt ihres Fachbereichs.
		
		\lines[4cm]
	\end{enumerate}
	
	
	\pagebreak
	Lösungsvorschläge zu Aufgabe 1:
	
	a)
	\begin{lstlisting}[language=sql]
SELECT *
FROM STUDENT
WHERE semester > (SELECT AVG(semester) FROM STUDENT);
	\end{lstlisting}
	
	b)
	\begin{lstlisting}[language=sql]
SELECT *
FROM KURS
WHERE ects = (SELECT MAX(ects) FROM KURS);
	\end{lstlisting}
	
	c)
	\begin{lstlisting}[language=sql]
SELECT l.lehrkraft_id, l.name
FROM LEHRKRAFT l
WHERE
	(SELECT AVG(k.ects) FROM KURS k
	WHERE k.lehrkraft_id = l.lehrkraft_id)
	>
	(SELECT AVG(ects) FROM KURS);
	\end{lstlisting}
	
	d)
	\begin{lstlisting}[language=sql]
SELECT f.fachbereich_id, f.name
FROM FACHBEREICH f
WHERE
	(SELECT MIN(b.note) FROM STUDENT s
	JOIN BELEGUNG b ON b.stud_id = s.stud_id
	WHERE s.fachbereich_id = f.fachbereich_id)
	=
	(SELECT MIN(note) FROM BELEGUNG);
	\end{lstlisting}
	
	e)
	\begin{lstlisting}[language=sql]
SELECT s.stud_id, s.name
FROM STUDENT s
WHERE
	(SELECT AVG(b.note) FROM BELEGUNG b
	WHERE b.stud_id = s.stud_id)
	<
	(SELECT AVG(b2.note)
	FROM BELEGUNG b2
	JOIN STUDENT s2 ON s2.stud_id = b2.stud_id
	WHERE s2.fachbereich_id = s.fachbereich_id);
	\end{lstlisting}
\end{document}