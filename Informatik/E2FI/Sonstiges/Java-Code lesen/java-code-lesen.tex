\documentclass[11pt, a4paper, oneside]{article}
\usepackage[ngerman]{babel}
\usepackage{worksheet}

\begin{document}
	\author{L. Bung}
	\title{Java-Code lesen}
	\subject{SAE}
	\class{E2FI}
	\maketitle
	
	Bei uns an der Schule wird in SAE ausschließlich Python programmiert.
	Im Hinblick auf die Teil-1-Prüfung (und auch allgemein!) kann es aber hilfreich sein, auch schon mal andere Programmiersprachen gesehen zu haben.
	
	\singletask{Summe der ersten $n$ natürlichen Zahlen}
	
	Schreiben Sie den folgenden Java-Code in Python:

	\begin{lstlisting}[language=java]
import java.util.Scanner;

public class Summe {
  public static void main(String[] args) {
    Scanner scanner = new Scanner(System.in);
    System.out.println("Gib eine Zahl ein:");
    int n = scanner.nextInt();
    int summe = 0;

    for (int i = 1; i <= n; i++) {
      summe += i;
    }

    System.out.println("Die Summe der ersten " + n + " Zahlen ist: " + summe);
  }
}
	\end{lstlisting}
	
	
	\singletask{Primzahl-Überprüfung}
	
	Schreiben Sie den folgenden Java-Code in Python:
	
	\begin{lstlisting}[language=java]
import java.util.Scanner;

public class Primzahl {
  public static void main(String[] args) {
    Scanner scanner = new Scanner(System.in);
    System.out.println("Gib eine Zahl ein:");
    int zahl = scanner.nextInt();

    boolean istPrim = true;

    for (int i = 2; i <= Math.sqrt(zahl); i++) {
      if (zahl % i == 0) {
        istPrim = false;
        break;
      }
    }

    if (istPrim && zahl > 1) {
      System.out.println(zahl + " ist eine Primzahl.");
    } else {
      System.out.println(zahl + " ist keine Primzahl.");
    }
  }
}
	\end{lstlisting}
	
	
	\singletask{Fibonacci-Zahlen berechnen}
	
	Schreiben Sie den folgenden Java-Code in Python:
	
	\begin{lstlisting}[language=java]
import java.util.Scanner;

public class Fibonacci {
  public static void main(String[] args) {
    Scanner scanner = new Scanner(System.in);
    System.out.println("Gib eine Zahl ein:");
    int n = scanner.nextInt();

    int a = 0, b = 1;

    System.out.print("Fibonacci-Zahlen bis " + n + ": ");

    for (int i = 0; i < n; i++) {
      System.out.print(a + " ");
      int temp = a;
      a = b;
      b = temp + b;
    }
  }
}
	\end{lstlisting}
\end{document}