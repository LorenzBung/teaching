\documentclass[11pt, a4paper, oneside]{article}
\usepackage[ngerman]{babel}
\usepackage{worksheet}

\begin{document}
	\author{L. Bung}
	\title{Wiederholung \hspace{5cm} Grundlagen Python}
	\subject{SAE}
	\class{E2FI}
	\maketitle
	
	Die folgenden Aufgaben haben mehrere Ziele:
	
	\begin{enumerate}
		\item Sie sollen durch die Bearbeitung Ihr Wissen zum Thema Python wieder auffrischen.
		\item Beim Lösen auftretende Schwierigkeiten sollen potenzielle Lücken identifizieren und Ihnen helfen, diese zeitnah zu schließen.
		\item Das Ergebnis Ihrer Arbeit liefert mir als Lehrkraft Rückmeldung darüber, mit welchem Vorwissen ich bei Ihnen rechnen kann.
	\end{enumerate}
	
	Das Arbeitsblatt ist so aufgebaut, dass die Aufgaben nacheinander schwieriger werden.
	Manchmal behandeln einzelne Aufgaben auch nur einzelne Themen.
	Falls Sie z. B. zu Stringoperationen nichts (mehr) wissen, können Sie diese Aufgabe einfach überspringen, mir dies aber bitte zurückmelden.
	
	\singletask{EVA-Prinzip und Variablen}
	
	\begin{enumerate}[label=\alph*)]
		\item Schreiben Sie ein Programm, das Ihren Namen und Ihr Alter in Variablen speichert und beides ausgibt.
		\item Schreiben Sie ein Programm, das zwei Werte von der Nutzereingabe einliest.
		Anschließend sollen Summe, Differenz, Produkt und Quotient der beiden Zahlen berechnet und in jeweils neuen Variablen gespeichert werden.
		Geben Sie zum Schluss die errechneten Werte schön formatiert aus.
		\item Berechnen Sie den Flächeninhalt eines Rechtecks anhand von Eingaben für Länge und Breite.
	\end{enumerate}
	
	\singletask{Verzweigungen}
	
	\begin{enumerate}[label=\alph*)]
		\item Schreiben Sie ein Programm, das eine eingegebene Zahl überprüft: ist sie gerade oder ungerade?
		\item Lassen Sie den Benutzer sein Alter eingeben und geben Sie je nach Alter eine Nachricht aus: ``Kind'', ``Jugendlicher'', ``Erwachsener''.
		\item Lassen Sie den Benutzer eine Zahl eingeben und geben Sie aus, ob sie positiv, negativ oder null ist.
	\end{enumerate}
	
	\pagebreak
	
	\singletask{Schleifen}
	
	\begin{enumerate}[label=\alph*)]
		\item Geben Sie die Zahlen von 1 bis 50 aus.
		\item Berechnen Sie die Summe aller Zahlen von 1 bis 100.
		\item Lassen Sie den Benutzer eine Zahl eingeben und geben Sie die kleine Einmaleins-Tabelle für diese Zahl aus (z. B. $7 \cdot 1$ bis $7 \cdot 10$).
		\item Lassen Sie den Benutzer wiederholt Zahlen eingeben, bis er \texttt{0} eingibt. Geben Sie dann die Summe aller eingegebenen Zahlen aus.
	\end{enumerate}
	
	\singletask{Listen}
	
	\begin{enumerate}[label=\alph*)]
		\item Legen Sie eine Liste mit 5 beliebigen Städten an und geben Sie sie aus.
		\item Lassen Sie den Benutzer 5 Zahlen eingeben. Speichern Sie die Werte in einer Liste und berechnen Sie den Durchschnitt.
		\item Suchen Sie das Maximum und Minimum in einer Liste ohne die eingebauten Funktionen \texttt{max()} und \texttt{min()}.
		\item Vertauschen Sie die Reihenfolge einer Liste (z. B. mithilfe einer Schleife).
	\end{enumerate}
	
	\singletask{Stringoperationen}
	
	\begin{enumerate}[label=\alph*)]
		\item Lassen Sie den Benutzer einen Satz eingeben und geben Sie die Anzahl der Wörter aus.
		\item Zählen Sie, wie oft ein bestimmter Buchstabe in einem String vorkommt.
		\item Lassen Sie einen Satz in Kleinbuchstaben umwandeln und prüfen Sie, ob er das Wort ``python'' enthält.
		\item Lassen Sie den Benutzer eine E-Mail-Adresse eingeben und trennen Sie sie in Benutzername und Domain auf.
	\end{enumerate}
	
	\singletask{Funktionen}
	
	\begin{enumerate}[label=\alph*)]
		\item Schreiben Sie eine Funktion \texttt{flaeche\_rechteck(l, b)}, die die Fläche eines Rechtecks berechnet und zurückgibt.
		\item Schreiben Sie eine Funktion, die prüft, ob eine Zahl eine Primzahl ist.
		\item Schreiben Sie eine Funktion, die eine Liste entgegennimmt und das größte Element zurückgibt.
		\item Schreiben Sie eine Funktion, die einen Satz bekommt und die Wörter alphabetisch sortiert zurückgibt.
		\item Schreiben Sie eine Funktion \texttt{woerter\_zaehlen(text)}, die ein Dictionary mit Worthäufigkeiten zurückgibt.
	\end{enumerate}
	
	\singletask{Sets}
	
	\begin{enumerate}[label=\alph*)]
		\item Legen Sie zwei Mengen mit Zahlen an und geben Sie die Schnittmenge, Vereinigung und Differenz aus.
		\item Lassen Sie den Benutzer eine Liste mit Zahlen eingeben. Entfernen Sie alle Duplikate mithilfe eines Sets.
		\item Zählen Sie, wie viele verschiedene Buchstaben in einem Satz vorkommen.
	\end{enumerate}
	
	\singletask{Dictionaries}
	
	\begin{enumerate}[label=\alph*)]
		\item Legen Sie ein Dictionary mit 5 Ländern und deren Hauptstädten an. Geben Sie es anschließend aus.
		\item Lassen Sie den Benutzer ein Land eingeben und geben Sie die Hauptstadt aus (falls vorhanden).
		\item Zählen Sie die Häufigkeit jedes Wortes in einem eingegebenen Satz.
		\item Erstellen Sie ein Dictionary, das Noten (1--6) zu verbalen Bewertungen (``sehr gut'', ``gut'', ...) zuordnet.
		Geben Sie für eine eingegebene Note die zugehörige verbale Bewertung aus.
	\end{enumerate}
	
	\singletask{Kombination der Konzepte}
	
	\begin{enumerate}[label=\alph*)]
		\item Schreiben Sie ein Programm, das den Benutzer nach beliebig vielen Namen fragt, bis dieser ``stop'' eingibt, und speicheren Sie die Namen in einer Liste.
		Geben Sie anschließend die Namen alphabetisch sortiert aus.
		\item Lassen Sie den Benutzer einen Satz eingeben und zählen Sie, wie oft jedes Wort vorkommt.
		Geben Sie das Ergebnis als Dictionary aus.
		\item Schreiben Sie ein Programm, das eine kleine Telefonliste mit einem Dictionary realisiert (Name $\mapsto$ Telefonnummer).
		Der Benutzer soll neue Einträge hinzufügen und nach Nummern suchen können.
		\item Erstellen Sie ein Vokabeltrainer-Programm.
		Speichern Sie dazu Wörter in einem Dictionary (Deutsch $\mapsto$ Englisch).
		Der Benutzer soll abgefragt werden, bis er eine bestimmte Anzahl richtig hat.
		\item Implementieren Sie ein Programm, das eine zufällige Zahl zwischen 1 und 100 wählt.
		Der Benutzer soll sie durch Eingaben erraten.
		Das Programm gibt Hinweise (``zu groß'', ``zu klein'') und zählt die Versuche.
	\end{enumerate}
\end{document}