\documentclass[11pt, a4paper, oneside]{article}
\usepackage[ngerman]{babel}
\usepackage{worksheet}

\begin{document}
	\author{L. Bung}
	\title{Datenbanken: Entity-Relationship-Model}
	\subject{SAE}
	\class{E2FI}
	\maketitle
	
	Bevor man anfängt, eine Datenbank mit Tabellen zu erstellen, muss man sich natürlich im klaren darüber sein, welche Struktur die Daten haben, die darin gespeichert werden sollen.
	Gerade bei großen Datenbanken ist es daher wichtig, sich vorher einen Plan zu machen und genau zu analysieren, wie die Datenbank aufgebaut sein muss.
	
	\hint{Entity-Relationship-Model (ERM)}{
		Das \emph{Entity-Relationship-Model (ERM)} dient dazu, die Struktur der Daten abzubilden, die sich in einer Datenbank befinden.
		Es setzt sich (wie der Name bereits andeutet) zusammen aus drei Komponenten:
		\begin{itemize}
			\item \emph{Entities}: Dies sind die Objekte, die modelliert werden sollen (beispielsweise \texttt{Kunde}).
			Entitäten werden als Rechteck dargestellt.
			\item \emph{Attribute}: Die Eigenschaften der Entitäten -- beim Beispiel \texttt{Kunde} wären das z.B. die Eigenschaften \texttt{Name} und \texttt{Geburtsdatum}, aber auch beispielsweise seine eindeutige \texttt{KundenID}.
			Attribute werden als Kreis bzw. Oval dargestellt und sind mit ihrer zugehörigen Entität verbunden.
			Primärschlüssel werden im ER-Modell unterstrichen.
			\item \emph{Relationships}: Die Beziehungen, in denen verschiedene Objekte miteinander stehen -- z.B. die Beziehung \texttt{gibt auf} zwischen den Entitäten \texttt{Kunde} und \texttt{Bestellung}.
			Relationships werden als Rauten dargestellt, die zwei Entitäten miteinander verbinden.
			An die Verbindungen schreibt man dazu, ob es sich um eine 1:1, 1:n oder n:m-Beziehung handelt.
		\end{itemize}
	}
	
	Ein einfaches Entity-Relationship-Model, in dem sowohl Entities, Attribute und Relationships vorkommen, sähe beispielsweise folgendermaßen aus:
	
	\begin{figure}[H]
		\centering
		\begin{tikzpicture}[node distance=3.5cm]
			\node[entity] (kunde) {Kunde};
			\node[attribute] (kundenid) [left of=kunde] {\underline{KundenID}} edge (kunde);
			\node[attribute] (name) [below left of=kunde] {Name} edge (kunde);
			\node[attribute] (gebdatum) [above left of=kunde] {GebDatum} edge (kunde);
			\node[relationship] (gibtauf) [right of=kunde] {gibt auf} edge node[above]{$1$} (kunde);
			\node[entity] (bestellung) [right of=gibtauf] {Bestellung} edge node[above]{$n$} (gibtauf);
			\node[attribute] (bestellungsid) [above right of=bestellung] {\underline{BestellungsID}} edge (bestellung);
			\node[attribute] (preis) [below right of=bestellung] {Preis} edge (bestellung);
		\end{tikzpicture}
	\end{figure}
	
	\pagebreak
	
	\singletask{Bibliothek}
	
	Eine Bibliothek möchte ihre Bestände erfassen.
	\begin{itemize}
		\item Es gibt Bibliotheken mit Name und Adresse.
		\item Es gibt Bücher mit ISBN, Titel und Autor.
	\end{itemize}
	Modellieren Sie, welche Bibliotheken welche Bücher besitzen.
	
	\boxarea[6cm]
	
	\singletask{Klassenbuch}
	
	Eine Schule möchte ihre Schüler und Klassen erfassen.
	\begin{itemize}
		\item Es gibt Schüler mit Name, Geburtsdatum und Adresse.
		\item Es gibt Klassen mit Klassenbezeichnung (z.B. ``7a'') und einem Klassenraum.
	\end{itemize}
	Modellieren Sie, welche Schüler in welchen Klassen sind.
	
	\boxarea[6cm]
	
	\pagebreak
	
	\singletask{Fluggesellschaft}
	
	Eine Fluggesellschaft möchte ihre Flugdaten verwalten.
	\begin{itemize}
		\item Es gibt Flugzeuge mit Kennung, Modell und Kapazität.
		\item Es gibt Flüge mit Flugnummer, Datum und Uhrzeit.
		\item Es gibt Passagiere mit Name und Geburtsdatum.
		\item Es gibt Flughäfen mit Code, Name und Stadt.
	\end{itemize}
	Modellieren Sie...
	\begin{itemize}
		\item ...mit welchem Flugzeug ein Flug durchgeführt wird
		\item ...welche Passagiere an einem Flug teilnehmen
		\item ...an welchen Flughäfen ein Flug startet und landet
	\end{itemize}
	
	\boxarea[10cm]
	
	\pagebreak
	
	\singletask{Musikstreaming}
	
	Eine Musikplattform möchte ihre Daten strukturieren.
	\begin{itemize}
		\item Es gibt Nutzer mit Name und Beitrittsdatum.
		\item Es gibt Songs mit Titel und Dauer.
		\item Es gibt Alben mit Titel und Erscheinungsjahr.
		\item Es gibt Künstler mit Namen.
		\item Es gibt Playlists mit Titel und Erstellungsdatum.
	\end{itemize}
	Modellieren Sie...
	\begin{itemize}
		\item ...welche Songs zu welchem Album gehören
		\item ...welche Alben von welchen Künstlern erstellt wurden
		\item ...welche Playlists von welchen Nutzern erstellt wurden
		\item ...welche Songs in Playlists enthalten sind
		\item ...welche Songs von welchen Nutzern angehört wurden
	\end{itemize}
	
	\boxarea[10cm]
	
	\pagebreak
	
	\bonustask{Umsetzung}
	
	Setzen Sie die modellierten Datenbanken mit SQL um:
	\begin{enumerate}[label=\alph*)]
		\item überlegen Sie sich geeignete Datentypen für die Attribute
		\item erstellen Sie die Tabellen mit SQL
		\item fügen Sie einige Beispieleinträge in die Tabellen ein
	\end{enumerate}
	
	\boxarea[18cm]
	
\end{document}