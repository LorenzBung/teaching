\documentclass[11pt, a4paper, oneside]{article}
\usepackage[ngerman]{babel}
\usepackage{worksheet}

\begin{document}
	\author{L. Bung}
	\title{Einführung Datenbanken}
	\subject{SAE}
	\class{E2FI}
	\maketitle
	
	\partnertask{Warum Datenbanken?}
	
	Stellen Sie sich vor, Sie entwickeln ein System zum Verwalten von Produkten.
	Es sollen verschiedene Produkte mit deren Artikelnummer, Preis, Lagerregal-Nummer, Gewicht und den Maßen (Länge und Breite) gespeichert werden.
	
	a) In welcher Form würden Sie diese Daten speichern?
	Diskutieren Sie mögliche Vor- und Nachteile Ihrer gewählten Form.
	
	\boxarea[6cm]
	
	b) Zusätzlich zu den Produkten selbst sollen auch Kunden im System gespeichert werden (mit Vor- und Nachnamen, Straße, Hausnummer, Postleitzahl, Stadt und Land sowie der Bankverbindung, bestehend aus IBAN und BIC).
	Können Sie diese Informationen in Ihrem Modell speichern, oder müssen Sie es verändern?
	
	\boxarea[6cm]
	
	c) Wie können Sie sicherstellen, dass Sie einen Kunden richtig identifizieren?
	Es könnten ja beispielsweise auch mehrere Kunden \emph{Thomas Müller} heißen.
	
	\boxarea[3cm]
	
	d) Wie könnten Sie nun auch noch die einzelnen Bestellungen der Kunden verwalten?
	Was für Informationen müssen Sie dafür speichern?
	
	\boxarea[3cm]
	
	\hint{Relationale Datenbanken}{
		Relationale Datenbanken verwalten die in Ihnen gespeicherten Daten in Tabellen, welche untereinander in Beziehung stehen.
		Eine solche Tabelle hat einen eindeutigen \textbf{Namen}, eine oder mehrere Spalten (\textbf{Attribute}) sowie eine beliebig große Anzahl an Zeilen (\textbf{Datensätzen}).
		Jedes Attribut hat dabei einen bestimmten \textbf{Datentyp} bzw. Wertebereich.
	}
	
	\task{Warum nicht einfach Excel verwenden?}
	
	Wenn es sich bei relationalen Datenbanken einfach um Tabellen handelt, könnte man doch einfach auch Excel-Tabellen verwenden, oder?
	Diskutieren Sie: Was spricht dagegen?
	
	\boxarea[6cm]
	
\end{document}