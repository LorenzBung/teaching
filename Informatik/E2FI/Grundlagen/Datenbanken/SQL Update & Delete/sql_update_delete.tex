\documentclass[11pt, a4paper, oneside]{article}
\usepackage[ngerman]{babel}
\usepackage{worksheet}

\begin{document}
	\author{L. Bung}
	\title{SQL: Update \& Delete}
	\subject{SAE}
	\class{E2FI}
	\maketitle
	
	\hint{Löschen von Werten aus einer Tabelle}{
		Bestimmte Datensätze können mit SQL aus einer Tabelle gelöscht werden, indem man das Keyword \texttt{DELETE} verwendet.
		In der Regel ist es sinnvoll, dies mit einem \texttt{WHERE}-Statement zu verbinden, da sonst alle Datensätze der Tabelle verloren gehen.
	}
	
	Der folgende Befehl löscht alle Kunden, die 'Max Weber' heißen und die Telefonnummer '+49 1234 567890' haben.
	
	\begin{lstlisting}[language=sql]
DELETE FROM Kunde WHERE Name = 'Max Weber' AND Telefonnummer = '+49 1234 567890';
	\end{lstlisting}
	
	\hint{Verändern von Datensätzen in der Tabelle}{
		Bereits existierende Datensätze können auch nachträglich noch verändert werden, indem die Keywords \texttt{UPDATE} und \texttt{SET} verwendet werden.
		Auch hier ist es meist notwendig, ein \texttt{WHERE}-Statement zu verwenden, da sonst alle Datensätze aktualisiert werden.
	}
	
	Der folgende Befehl setzt die Postleitzahl aller Kunden auf '79098', die in Freiburg wohnen und 'Maria Bauer' heißen.
	\begin{lstlisting}[language=sql]
UPDATE Kunde SET plz = '79098' WHERE Ort = 'Freiburg' AND Name = 'Maria Bauer';
	\end{lstlisting}
	
	\pagebreak
	
	\singletask{Immobilienverwaltung}
	
	Eine Immobiliengesellschaft verwaltet Wohnungen und speichert dazu jeweils eine ID, Anzahl der Zimmer und Mietkosten ab und ob die Wohnung belegt ist.
	In einer Immobilie können sich mehrere Wohnungen befinden.
	Zu den Immobilien werden eine ID, die Straße, Hausnummer, Stadt und Postleitzahl abgespeichert.
	
	a) Erstellen Sie ein ER-Modell der Situation.
	
	\boxarea[7cm]
	
	b) Ändern Sie die Mietkosten der Wohnung mit der ID 101 auf 800.
	
	\lines[2cm]
	
	c) Setzen Sie die Mietkosten aller Wohnungen, die 2 Zimmer haben und aktuell unter 800 kosten auf 1000.
	
	\lines[2cm]
	
	\pagebreak
	
	d) Ändern Sie die Mietkosten aller Wohnungen, deren Anzahl Zimmer zwischen 2 und 4 liegt und die aktuell leer stehen, auf 950.
	
	\lines[2cm]
	
	e) Setzen Sie die Postleitzahl aller Immobilien auf 79098, die in Freiburg sind und in der Salzstraße, Bismarckallee oder Friedrichring liegen.
	
	\lines[2cm]
	
	f) Löschen Sie die Wohnung mit der ID 205.
	
	\lines [2cm]
	
	g) Löschen Sie alle Wohnungen, die leer stehen und zwischen 3 und 5 Zimmern haben.
	
	\lines[2cm]
	
	h) Löschen Sie alle Immobilien, die nicht in Stuttgart, Heidelberg, Tübingen oder Freiburg liegen und deren Postleitzahl mit 7 beginnt.
	
	\lines[2cm]
\end{document}