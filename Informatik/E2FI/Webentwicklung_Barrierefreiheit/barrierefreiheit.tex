\documentclass[11pt, a4paper]{article}

\usepackage{worksheet}

\begin{document}
	\author{L. Bung}
	\title{Webentwicklung: \hspace{10cm}Barrierefreiheit}
	\subject{SAE}
	\class{E2FI}
	\maketitle
	
	\grouptask{Probleme und Einschränkungen im Internet}
	
	a) Sammeln Sie gemeinsam mögliche Probleme und Einschränkungen für Nutzer, die bei der Webentwicklung berücksichtigt werden müssen.
	
	\lines[4cm]
	
	b) Ordnen Sie die gesammelten Aspekte den vier Kernprinzipien der WCAG\footnote{\url{https://www.w3.org/TR/WCAG21/}} zu -- Wahrnehmbarkeit, Bedienbarkeit, Verständlichkeit und Robustheit.
	
	\boxarea[10cm]
	
	\hint{Merke}{Die Standards für Barrierefreiheit im Internet werden durch die \textit{Web Content Accessibility Guidelines (WCAG)} festgelegt.
		Sie beinhalten vier Kernprinzipien:
		\textbf{Wahrnehmbarkeit}, \textbf{Bedienbarkeit}, \textbf{Verständlichkeit} und \textbf{Robustheit}.}
	
	\singletask{Umsetzung bei der Webentwicklung}
	
	In Moodle finden Sie einen Ordner mit Bausteinen einer Webseite:
	
	\begin{enumerate}
		\item Ein Teil einer Seite, die das Team einer Firma vorstellt (1 HTML- und 1 JPG-Datei)
		\item Eine Navigationsleiste, in welcher verschiedene Seiten verlinkt sind (1 HTML-Datei)
		\item Die Beschreibung der Dienstleistungen der Firma (2 HTML-Dateien)
	\end{enumerate}

	Analysieren Sie die Webseiten in Bezug auf die vier Kernprinzipien der Barrierefreiheit.
	Korrigieren Sie anschließend die Probleme, die Sie gefunden haben.
	
	\bonustask{Barrierefreiheit auf bekannten Webseiten}
	
	Besuchen Sie eine bekannte Webseite Ihrer Wahl (z.B. eine Nachrichtenseite, Onlineshop usw).
	Testen Sie die Seite ausgiebig auf ihre Barrierefreiheit.
	Notieren Sie sich alle Barrieren, die Ihnen auffallen und beschreiben Sie konkrete Verbesserungsvorschläge.
	
	\lines[4cm]
\end{document}