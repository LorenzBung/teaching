\documentclass[11pt, a4paper]{article}

\usepackage{worksheet}

\begin{document}
	\author{L. Bung}
	\title{Barrierefreiheit \hspace{10cm}(Lösungsvorschlag)}
	\subject{SAE}
	\class{E2FI}
	\maketitle
	
	\grouptask{Probleme und Einschränkungen im Internet}
	
	\textbf{Wahrnehmbarkeit}: Kein alt-Tag bei Bildern, Schlechter Kontrast (z.B. bei Buttons), Skalierbarkeit der Webseite (Verwendung von Pixel-Einheiten statt relativen Größen)
	
	\textbf{Bedienbarkeit}: Dropdown-Menüs nur bei Hover sichtbar, Fokussierte Elemente nicht sichtbar, Elemente nur mit Maus nutzbar
	
	\textbf{Verständlichkeit}: Verwendung komplizierter Sprache, Linktexte ohne Aussagekraft (``hier klicken''), unklare Fehlermeldungen
	
	\textbf{Robustheit}: Falsch verwendete HTML-Tags (z.B. \texttt{<table>}), Verwendung veralteter Tags (z.B. \texttt{<b>}), fehlerhaftes HTML
	
	\singletask{Umsetzung bei der Webentwicklung}
	
	\lstinputlisting[language=html, caption={Lösungsvorschlag zu Aufgabe 2.1}, captionpos=b]{Aufgabe2/muster/aufgabe2-1-muster.html}

	\lstinputlisting[language=html, caption={Lösungsvorschlag zu Aufgabe 2.2}, captionpos=b]{Aufgabe2/muster/aufgabe2-2-muster.html}
	
	\lstinputlisting[language=html, caption={Lösungsvorschlag zu Aufgabe 2.3}, captionpos=b]{Aufgabe2/muster/aufgabe2-3-muster.html}
	
	\lstinputlisting[language=html, caption={Lösungsvorschlag zur in Aufgabe 2.3 verlinkten HTML-Seite}, captionpos=b]{Aufgabe2/muster/leistungen-muster.html}
\end{document}