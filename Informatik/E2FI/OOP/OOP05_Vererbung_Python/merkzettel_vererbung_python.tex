\documentclass[11pt, a4paper, oneside]{article}
\usepackage{worksheet}

\begin{document}
	\author{L. Bung}
	\title{Merkzettel \hspace{10cm} Vererbung in Python}
	\subject{SAE}
	\maketitle
	
	\lstinputlisting[language=python]{vererbung.py}
	
	\hint{Vererbung in Python}{
	Um eine Vererbungsrelation in Python anzugeben, wird der Name der Superklasse bei der Definition der Subklasse in Klammern geschrieben.
	Im obigen Beispiel erbt also die Klasse \texttt{Auto} von der Klasse \texttt{Fahrzeug}.
	
	Objekte der Subklasse können auf Attribute und Methoden zugreifen, die in der Superklasse definiert wurden.
	Beispielsweise kann \texttt{auto1.fahren()} aufgerufen werden, obwohl die Methode \texttt{fahren()} in der Klasse \texttt{Fahrzeug} definiert wurde.
	}

	\hint{Aufruf von Methoden der Superklasse}{
	Mit \texttt{super()} können Methoden der Superklasse aufgerufen werden.
	Dies ist insbesondere bei den Konstruktoren (\texttt{\_\_init\_\_()}-Methoden) hilfreich, um doppelten Code zu sparen.
	
	In Zeile 8 müsste sonst das Attribut \texttt{geschwindigkeit} manuell gesetzt werden.
	Bei vielen Parametern wäre das sehr aufwändig.
	}
\end{document}
