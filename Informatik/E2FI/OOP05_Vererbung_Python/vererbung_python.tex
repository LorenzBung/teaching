\documentclass[11pt, a4paper, oneside]{article}
\usepackage{worksheet}

\begin{document}
	
	\makeheader{SAE}{OOP: Vererbung (Python)}
	
	\singletask{Gartencenter-Produktverwaltung in Python}
	
	Nachdem wir in der letzten Stunde gemeinsam ein Softwarekonzept für die Produktverwaltung des Gartencenters entworfen haben, soll dieses nun implementiert werden.
	
	\begin{figure}[h]
		\centering
		\begin{tikzpicture}
			\begin{class}[text width=4cm, text height=.3cm]{Produkt}{5,5}
				\attribute{produktnummer: int}
				\attribute{bezeichnung: str}
				\attribute{preis: float}
				\operation{verkaufen()}
			\end{class}
			\begin{class}[text width=4cm, text height=.3cm]{Pflanze}{0,0}
				\inherit{Produkt}
				\attribute{gattung: str}
				\attribute{hoehe: int}
				\operation{einpflanzen()}
			\end{class}
			\begin{class}[text width=4cm, text height=.3cm]{Blumentopf}{5,0}
				\inherit{Produkt}
				\attribute{material: str}
				\attribute{volumen: int}
				\operation{befuellen()}
			\end{class}
			\begin{class}[text width=4cm, text height=.3cm]{Werkzeug}{10,0}
				\inherit{Produkt}
				\attribute{istElektrisch: bool}
			\end{class}
		\end{tikzpicture}
	\end{figure}

	Setzen Sie das obenstehende UML-Diagramm in Python um.
	Bilden Sie insbesondere die Vererbungsstrukturen korrekt ab.
	
	\partnertask{Richtung der Vererbung}
	
	Betrachten Sie den folgenden Code.
	
	\begin{lstlisting}[language=python]
class Laptop:
	def leistung(self):
		return "Alltagstauglich"
class GamingLaptop(Laptop):
	def leistung(self):
		return "Sehr leistungsstark"
	\end{lstlisting}

	a) Verändern Sie den Code, sodass die Richtung der Vererbung umgedreht wird.
	
	b) Diskutieren Sie, welche Variante mehr Sinn ergibt.
	
	\pagebreak
	
	\bonustask{Mehrfachvererbung}
	
	Am Ende der letzten Stunde haben wir kurz über die Probleme gesprochen, die bei mehrfacher Vererbung entstehen können (Beispiel: \texttt{Wasserflugzeug} als Subklasse von \texttt{Flugzeug} und \texttt{Boot}).
	
	a) Recherchieren Sie, wie in Python mit Mehrfachvererbung umgegangen wird.
		
	\lines[3cm]
		
	b) Setzen Sie das oben beschriebene Beispiel in Python um und halten Sie Ihre Beobachtungen fest.
		
	\lines[3cm]
	
\end{document}
