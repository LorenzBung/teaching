\documentclass[11pt, a4paper, oneside]{article}
\usepackage{worksheet}
\usepackage{lipsum}

\begin{document}
	\author{L. Bung}
	\title{Subnetz-Rechner}
	\subject{ITS}
	\maketitle
	
	Bei den folgenden Aufgaben geht es darum, die Subnetze automatisiert berechnen zu lassen.
	Schreiben Sie dazu ein Python-Programm, welches aus der CIDR-Notation die Netzwerk- und Broadcast-Adresse berechnet und ausgibt.
	
	\grouptask{IPv4-Subnetz berechnen}
	
	Berechnen Sie die Netzwerk- und Broadcastadresse: \texttt{192.168.178.21/24}
	
	\boxarea[7cm]
	
	\singletask{Nutzereingabe}
	
	Schreiben Sie eine Funktion, die eine IPv4-Adresse mit Subnetz in CIDR-Notation einliest.
	Spalten Sie die Eingabe auf, so dass die vier Blöcke der IP-Adresse sowie das Suffix nutzbar werden.
	
	\hint{Hinweis}{Mit der \texttt{split()}-Funktion können Strings in Python geteilt werden.
		Beispielsweise liefert \texttt{"1 2 3 4".split(' ')} die Liste \texttt{['1', '2', '3', '4']}.}
	
	\singletask{Binärkonvertierung}
	
	Schreiben Sie eine Funktion, welche die (dezimale) IP-Adresse ins Binärsystem konvertiert.
	
	\hint{Hinweis}{Sie können die bereits vorgegebene Funktion \texttt{decimal\_to\_binary()} verwenden.
	Diese Funktion nimmt eine Dezimalzahl $0 \leq n \leq 255$ und wandelt diese in eine 8-stellige Binärzahl um.}
	
	\singletask{Netz- und Hostanteil}
	
	Schreiben Sie eine Funktion, welche die zuvor berechnete binäre IP-Adresse in Netz- und Hostanteil spaltet.
	
	\singletask{Berechnung der Netzwerkadresse}
	
	Schreiben Sie eine Funktion, welche die (binäre) Netzwerkadresse berechnet.
	Schreiben Sie anschließend eine weitere Funktion, welche die Netzwerkadresse im Dezimalformat ausgibt.
	
	\hint{Hinweis}{Die vorgegebene Funktion \texttt{binary\_to\_decimal()} nimmt eine Binärzahl als String und gibt diese als Dezimalzahl zurück.}
	
	\singletask{Berechnung der Broadcastadresse}
	
	Schreiben Sie nun eine Funktion, welche die Broadcastadresse jeweils binär und dezimal zu berechnet und zurückgibt.
	Die Ergebnisse der vorherigen Aufgabe könnten hierfür nützlich sein.
	
	\bonustask{Validierung und Tests}
	
	Verifizieren Sie, dass es sich bei der Nutzereingabe um korrekte Werte handelt.
	Überprüfen Sie die CIDR-Notation: Ist die IPv4-Adresse gültig? Ist das Suffix richtig? Wurden korrekte Trennzeichen verwendet?
	
	\bonustask{IPv6-Unterstützung}
	
	Bisher kann das Programm nur mit IPv4-Adressen rechnen.
	Erweitern Sie es, so dass für IPv6-Adressen die erste und letzte Adresse im Subnetz berechnet werden können.
	
	\warning{Achtung}{Bei IPv6-Adressen gibt es weder Netzwerk- noch Broadcastadresse.}
\end{document}
